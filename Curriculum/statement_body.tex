\section{Statement of research interest}
\cventry{\textbf{}}{}{}{}{
After the observation of the Higgs boson, one of the main priorities for Run II of the LHC is the exploration of the energy frontier. No significant excess has been observed so far with the current Run II dataset, and stringent limits have been set on simplified models. 
The motivation for some form of new physics is still strong but it is obvious that it is not manifested in the vanilla signatures that were the main focus of attention. 
At this stage two different approaches can and should be pursued. One on side, searches in final states with low cross section or challenging signatures will become more relevant with the growing dataset. On the other side, precision measurements can provide a ground where deviations from the SM prediction can shed further light on the presence or absence of new physics.
}{}{}
% Higgs to bb
\cventry{\textbf{H $\bm{\rightarrow}$ bb}}{}{}{}{
%\cventry{\textbf{}}{}{}{}{
The discovery of the Higgs boson was a milestone in the LHC program, and although so far all the measurements are compatible with the SM prediction, most of its properties have not been measured yet. 
In particular the decay of the Higgs boson to b-quarks has not been observed, although it is the decay with highest branching ratio. The associated production of the Higgs with a vector boson is close to achieving the first evidence for such decay, and will be a key analysis to establish and measure the decay to b-quarks. Beside the intrinsic relevance of the VH channel, it is also a sensitive channel in which to look for BSM signals, for example in electroweak supersymmetry decays. This is an area where I would like to contribute, and where dedicated and detailed work can make a large impact towards a first observation.
}{}{}
% V+bb
\cventry{\textbf{V+bb}}{}{}{}{
%\cventry{\textbf{}}{}{}{}{
The large increase in energy and luminosity has been a great benefit to most analyses in ATLAS. However, the relative slow increase that is foreseen in the next years will demand an improvement in the understanding of background modeling and systematics in order to surpass significantly the current level of sensitivity. In particular the precise measurement of V+bb will be a key ingredient towards the observation of VH. Measurements of vector boson in association with b-quarks have been performed in Run 1. However, they don't profit from the increased beam energy available in Run 2 and with little focus on the collinear regime, where the two b-quarks are close and can merge into a single jet. This regime is both challenging and very interesting since MC generators are usually not able to model it correctly.
}{}{}
% gluon to bb
\cventry{\textbf{gluon $\rightarrow$ bb}}{}{}{}{
%\cventry{\textbf{}}{}{}{}{
The production of b-quarks from gluon splitting is a background faced by almost every analysis requiring b-jets in the final state, and the ability to identify such splittings would immediately improve many searches and measurements. Several implementations for such a tagger have been designed, however focusing mostly on the high-p$_{\text{T}}$ regime or specifically on Higgs decays. Given the multiple applications, I would like to work towards the development of a dedicated gluon to bb tagger, focusing on the low and intermediate p$_{\text{T}}$  regime. Any improvement in gluon identification will directly improve the aforementioned analyses, and can also be of wider interest to the whole collaboration. 
}{}{}