\setlength{\parskip}{4pt}

\section{Statement of research interest}

%Warwick
%   Vcb through top decay
%   Higgs in gammagamma and bb states
%   fragmentation analysis in minimum bias events
%   searches for new Higgs bosons, including a long-lived A decaying to bb
%   high level trigger for high luminosity running
%   quality control and construction of strip modules for the Silicon Tracker upgrade.

%%%%%%%%% Intro and hierarchy probelm
%\cvitem{}{
\indent After the successful observation of the Higgs boson in Run 1, the increased energy of Run 2 has provided the LHC experiments with an unprecedented dataset to explore the energy frontier. No significant excess has been observed so far, and stringent limits have been set on simplified models.
The motivation for some form of new physics is still strong, but it is obvious that it is not manifested in the vanilla signatures that were the main focus of attention.
The Higgs boson mass has been measured at the electroweak scale, and the corresponding \emph{hierarchy problem} remains an unsolved question that I would like to address.
%}

%%%%%%%%% Natural supersymmetry and RPV
%\cvitem{}{
\hspace{20pt}
Natural supersymmetry remains a prime candidate to solve the hierarchy problem, and in particular R-parity violating (RPV) supersymmetry models can feature naturally light mass spectra with only weak exclusion limits. One of the key ingredients in natural supersymmetry is the presence of light higgsinos. Searches for higgsino production in final states with missing energy are well established at the LHC, and will continue progressing. However, higgsino searches without missing energy, as predicted by RPV, are largely uncovered. My goal is to develop a programme to cover the natural region of light higgsinos, including the most challenging decays. While the motivation to develop such programme originates as a search for higgsinos, the final states to be covered are sensitive to a large variety of BSM models.%}
%Ewk production slepton-mediated > 1 TeV

%%%%%%%%% multi-lepton multi-bjet and LQD
\hspace{20pt}I have recently developed an analysis targeting higgsino production with RPV decay via a baryon-number-violating coupling $UDD$, exploiting the final state with at least one lepton and many $b$-jets ($\geq 1 \ell, \geq 3$ b-jets).
This search provides the first limits on hadronic higgsino decays since LEP. I would like to extend the analysis into the multi-leptons multi-$b$-jets final state ($\geq 3 \ell, \geq 3$ b-jets), in order to target also effectively decays via the lepton-number-violating coupling $LQD$, which is so far uncovered. 
Supersymmetric models with this coupling have received renewed interest recently given their potential to explain the $R_K$, $R_K^*$, and muon $g-2$ anomalies.
It is worth highlighting that searches in multi-lepton and multi-$b$-jets final states are not exclusive to RPV supersymmetry. Other models such as stealth supersymmetry or longer decay chains within the electroweak superpartners also predict similar final states, as well as non-supersymmetric models such as additional scalars or pseudoscalars in 2HDMs or vector-like quarks.
%A common aspect of these models is the possibility for higgsinos to decay via multiple vector bosons and higgs bosons or additional scalars, depleting the final state from \met. As part of the program, a search for b\=b resonances in this final state would tackle these kind of models. 
%Therefore, light higgsinos with such challenging decays have not been tested yet, and could also be accessed with the multi-lepton multi-$b$-jet final state.
%Therefore, the multi-lepton multi-$b$-jet final state can be used to target a large variety of models that can populate this challenging final state.


%%%%%%%%% long-lived 
\hspace{20pt}In order to achieve an exhaustive coverage of the higgsino landscape, a possibility that should not be neglected is that the higgsinos (or other BSM particles) could be long-lived. 
Searches for displaced decays, such as displaced vertices, have been conducted at the LHC, however focusing mostly on strong production with high-mass vertices.
Searches for displaced decay products from low-mass scalars also exist, but rely on the associated production with a vector boson, which is not possible in the case of higgsinos or new pseudoscalars.
In addition, the background levels increase dramatically when attempting to adapt current searches towards vertices with lower masses and lower number of tracks, as required to have acceptance to lower mass scales.
Therefore, I would like to pursue a programme for electroweak-scale displaced decays, leading to displaced vertices. 
%And in particular, focusing on processes such as higgsino production, that do not benefit from the possibility of associated production with a vector boson. 

%%%%%%%% DV with soft muons
\hspace{20pt}From an experimental point of view, one of the main handles to identify the signals previously discussed was the presence of a high number of $b$-jets. This feature is seemingly lost when considering displaced decays. One possibility to recover it however, is to exploit B-hadron decays to muons. By requiring a displaced vertex with one or more muons attached, the backgrounds can be massively reduced, enabling a search for displaced vertices with lower masses and lower number of tracks. Other interesting signals that can be accessed lowering the track multiplicity requirements are for example decays of new pseudoscalars to two displaced taus, which are so far uncovered. The low production cross section and additional penalty from the muonic branching ratio of B-hadrons or taus requires a large recorded dataset, but the achieved purity can be excellent, making this a suitable target for the upcoming LHC runs. 

%%%%%%%% trigger
\hspace{20pt}One of the main challenges of the described searches for long-lived particles is the absence of an effective trigger. The final state contains no significant \met, the low mass scale that is targeted is significantly below the hadronic trigger thresholds, and possible leptons in the decay are displaced, rendering the usual lepton triggers inefficient. Many searches rely on the associated production with a vector boson to trigger. However, this option is not present in the case of higgsino production or new pseudoscalars, and other trigger strategies need to be developed.
Two different approaches can and should be pursued. One on side, triggering directly on the displaced decay products can provide a high efficiency at the cost of some model-dependence. In particular I would like to develop a trigger for displaced vertices with additional displaced leptons in the vertex. The leptons can originate either from the BSM particle, top quarks, or b-quarks in the decay. The lepton provides handles to reduce the rate, and defines a region in the detector where the displaced vertex reconstruction can be run. Limiting therefore the rate and area where the CPU-intensive reconstruction of large-radius tracks has to run.
The second approach to triggering would be to rely on the products of some associated production, such as vector-boson-fusion (VBF) which can be exploited for this purpose. Triggering on a high-mass dijet system with large separation in $\eta$ would allow to reconstruct and trigger also on displaced vertices without leptons, opening up possibilities to target also low-mass signals without leptons or heavy-flavour quarks. Inclusive VBF triggers exist already, but with too high thresholds to be effective in targeting low-mass signals.

\hspace{20pt}Both of the proposed searches explore final states where the prediction of Monte-Carlo (MC) generators is not reliable and data-driven techniques are necessary. However, data-driven techniques have usually strong limitations in their applicability, not being able to model the full dimensionality of the final state, or leading to very large uncertainties. I am interested in exploring and developing new background estimation methods based on combinations of machine-learning and data-driven techniques. The estimation of density ratios via neural networks to provide a fully data-driven estimation have been used in the past. However, fully new techniques need to be developed for the challenging cases where a mixture of processes contribute, the learned reweighting has to be extrapolated in phase-space rather then interpolated, or the data statistics is too low to provide a training sample. These three challenges affect most of the analyses at LHC and new methods could provide a ground-breaking tool for LHC searches. 


%%%%%%%Conclusion

\hspace{20pt}As discussed, I consider that a programme of searches for higgsinos with challenging decays is needed to cover as exhaustively as possible one of the key predictions of natural supersymmetry. The proposed searches can be designed with minimal model dependence, and can be powerful probes of other BSM models such as 2HDMs. Extending such searches into the regime of long-lived decays is difficult but certainly possible. Dedicated triggers will need to be developed to effectively record such events. In both prompt and displaced searches, the lepton and $b$-jet multiplicities and identification criteria offer a continuous trade-off between purity and available statistics. Therefore proving the flexibility to fully exploit the increase in recorded dataset during the next years.