\section{Diversity statement}

\cvitem{}{
I am half-Spanish, half-German, living in France and working in Switzerland, in an international organization such as CERN with scientist of more than 100 nationalities.
With such upbringing I have had the pleasure of making friends and working with people from the most diverse possible backgrounds.
The working and mentoring experience I gained through the years have taught me the challenges and opportunities of interacting with a multicultural team.
It has also made me acutely aware of the bias and discrimination in our field, in most cases subconscious but nonetheless harmful.
}
\cvitem{}{
One of the challenges that I faced myself was in the selection of students to mentor in the CERN summer student program. I was aware of studies showing
systemic bias in the evaluation of CVs, where the same application was ranked on average lower if the name indicated a female candidate.
Therefore I asked a colleague to blur names and photographs in all the CVs that I would review to avoid any bias.
I also participated in programs to introduce individual students from Swiss high-schools to general physics research. Given my multicultural background I requested
to give priority to foreign students in the high school. I am happy to say that after our week together she was thrilled and enthusiastic about pursuing a career in physics research.
}

\cvitem{}{
I am well aware that being a scientist or researcher does not mean just being successful in
research. At the same time, one should be excellent in his/her interactions with the community
and the students, in his/her role to lead the academic society and in responsibilities to
transform the community.
}
\cvitem{}{
In conclusion, I believe academia must strive to expand diversity with a more inclusive approach, welcoming
and embracing different socioeconomic, ethnic, gender groups, and actively create a broader pool of thought
processes and worldviews.
}
