\section{List of publications}
\cvitem{}{Highlighted in bold are the three publications that I attach as the most significant ones.}

\cvitem{Run~1: ttH(bb)}{
I was one of the main analysers in the first ATLAS analysis targeting the \ttH\ production mode, contributing to the development of the selection, fitting strategy and background estimation. In particular the challenging \ttbb\ background modelling and associated systematic uncertainties, where a collaboration with theorists was started. I studied the \ttbb\ modelling for the first time at NLO, and integrated the dedicated calculation in the inclusive \ttbar+jets sample via a multi-dimensional reweighting.
\begin{itemize}
\item Search for the Standard Model Higgs boson produced in association with top quarks in proton-proton collisions at $\sqrt{s}=7$ TeV using the ATLAS detector~\cite{ATLAS-CONF-2012-135}.
\item \textbf{Search for the Standard Model Higgs boson produced in association with top quarks and decaying into $b\bar{b}$ in pp collisions at $\sqrt{s}=8$ TeV with the ATLAS detector}~\cite{HIGG-2013-27}. 
\item Measurements of the Higgs boson production and decay rates and coupling strengths using pp  collision data at $\sqrt{s}=7$ and 8 TeV in the ATLAS experiment~\cite{HIGG-2014-06}.
\item Measurements of the Higgs boson production and decay rates and constraints on its couplings from a combined ATLAS and CMS analysis of the LHC pp collision data at $\sqrt{s}=7$ and 8 TeV~\cite{HIGG-2015-07}.
\end{itemize}
}

\cvitem{Run~1: BSM ttbb}{
I developed searches for fermionic and bosonic top partners in the \ttbb\ final states. Both searches involved a full redesign of the previous \ttH(bb) analysis, tailoring the selection and fitted variables to the considered signals. For both analyses I was the main (and only) analyser, together with my supervisor.
\begin{itemize}
\item Search for production of vector-like quark pairs and of four top quarks in the lepton-plus-jets final state in pp collisions at $\sqrt{s}=8$ TeV with the ATLAS detector~\cite{EXOT-2013-18}.
\item ATLAS Run 1 searches for direct pair production of third-generation squarks at the Large Hadron Collider~\cite{SUSY-2014-07}.
\end{itemize}
}

\cvitem{Tile calorimeter}{
I worked in the characterisation of the Tile calorimeter timing performance and calibration with muons from collision events.
\begin{itemize}
\item Operation and performance of the ATLAS Tile Calorimeter in Run 1~\cite{TCAL-2017-01}.
\end{itemize}
}

\cvitem{Run~2: RPC SUSY}{
I was analysis contact in the search for top squarks in the single-lepton final state, leading a group of around 15 people. Redesigned the background estimation methods to reduce the reliance on MC simulation, and developed new selections to further suppress backgrounds. I also defined new signal benchmarks, in order explore more comprehensively challenging models with low stop masses.
\begin{itemize}
\item Search for top squarks in final states with one isolated lepton, jets, and missing transverse momentum in $\sqrt{s}=13$ TeV pp collisions with the ATLAS detector~\cite{SUSY-2015-02}.
\item Search for top squarks in final states with one isolated lepton, jets, and missing transverse momentum in $\sqrt{s}=13$ TeV pp collisions with the ATLAS detector~\cite{ATLAS-CONF-2016-050}. 
\item \textbf{Search for top-squark pair production in final states with one lepton, jets, and missing transverse momentum using 36 fb$^{-1}$ of $\sqrt{s}=13$ TeV pp collision data with the ATLAS detector}~\cite{SUSY-2016-16}.
\end{itemize}
}

\cvitem{Run~2: RPV SUSY}{
I co-designed with another CERN fellow a fully new analysis targeting the final state of a lepton plus many jets (up to 15 jets and 4 $b$-jets), which was previously uncovered. The main challenge was the background estimation at these extreme multiplicities for which we developed fully new data-driven methods. I am also paper editor for the full Run 2 paper. Due to its wide applicability I also contributed to its reinterpretation in four-top models, and SUSY models with displaced decays, where I was also CONF editor.
\begin{itemize}
\item Search for new phenomena in a lepton plus high jet multiplicity final state with the ATLAS experiment using $\sqrt{s}=13$ TeV proton-proton collision data~\cite{SUSY-2016-11}.
\item \textbf{Search for R-parity violating supersymmetry in a leptons plus high jet multiplicity final state with the ATLAS experiment using 139 fb$^{-1}$ of $\sqrt{s}=13$ TeV proton--proton collision data}~\cite{RPV1L} (in approval).
\item Constraints on mediator-based dark matter and scalar dark energy models using $\sqrt{s}=13$ TeV pp collision data collected by the ATLAS detector~\cite{EXOT-2017-32}.
\item Reinterpretation of searches for supersymmetry in models with variable R-parity-violating coupling strength and long-lived R-hadrons~\cite{ATLAS-CONF-2018-003}.
\end{itemize}
}


\cvitem{Trigger}{
I was editor of the 2017 trigger menu PUB note, and coordinator and supervisor of the 2018 trigger menu PUB note. I also was in charge of the design of the Run 3 trigger menu.
\begin{itemize}
\item Trigger Menu in 2017~\cite{ATL-DAQ-PUB-2019-001}.
\item Trigger Menu in 2018~\cite{ATL-DAQ-PUB-2018-002}.
\item Run 3 trigger menu design~\cite{Run3menu}.
\end{itemize}
}

