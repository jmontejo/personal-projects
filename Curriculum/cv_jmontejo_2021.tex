
\documentclass[11pt,a4paper,sans]{moderncv}        % possible options include font size ('10pt', '11pt' and '12pt'), paper size ('a4paper', 'letterpaper', 'a5paper', 'legalpaper', 'executivepaper' and 'landscape') and font family ('sans' and 'roman')

% moderncv themes
\moderncvstyle{classic}                            % style options are 'casual' (default), 'classic', 'oldstyle' and 'banking'
\moderncvcolor{blue}                               % color options 'blue' (default), 'orange', 'green', 'red', 'purple', 'grey' and 'black'
%\renewcommand{\familydefault}{\sfdefault}         % to set the default font; use '\sfdefault' for the default sans serif font, '\rmdefault' for the default roman one, or any tex font name
%\nopagenumbers{}                                  % uncomment to suppress automatic page numbering for CVs longer than one page


\makeatletter
\renewcommand*{\bibliographyitemlabel}{\@biblabel{\arabic{enumiv}}}
\makeatother

% adjust the page margins
\usepackage[scale=0.76]{geometry}
\setlength{\hintscolumnwidth}{2.05cm}                % if you want to change the width of the column with the dates
%\setlength{\makecvtitlenamewidth}{10cm}           % for the 'classic' style, if you want to force the width allocated to your name and avoid line breaks. be careful though, the length is normally calculated to avoid any overlap with your personal info; use this at your own typographical risks...

%%%%%%%%%%%%%% Nasty hack to avoid whitespace before the bib
\makeatletter
\renewenvironment{thebibliography}[1]{%
%     \section*{\refname}%
%      \@mkboth{\MakeUppercase\refname}{\MakeUppercase\refname}%
      \list{\@biblabel{\@arabic\c@enumiv}}%
           {\settowidth\labelwidth{\@biblabel{#1}}%
            \leftmargin\hintscolumnwidth
            \advance\leftmargin\labelsep
            \@openbib@code
            \usecounter{enumiv}%
            \let\p@enumiv\@empty
            \renewcommand\theenumiv{\@arabic\c@enumiv}}%
      \sloppy
      \clubpenalty4000
      \@clubpenalty \clubpenalty
      \widowpenalty4000%
      \sfcode`\.\@m}
     {\def\@noitemerr
       {\@latex@warning{Empty `thebibliography' environment}}%
      \endlist}
\makeatother

\usepackage[normalem]{ulem}
\usepackage{color}
\usepackage{bm}
\usepackage{amstext}
\usepackage{amssymb}
\newcommand{\coloredLink}[2]{\textcolor{blue}{\href{#1}{#2}}}
\usepackage{tikzpagenodes}


\newcommand\ttbb{\ensuremath{t\bar{t}b\bar{b}}}
\newcommand\ttbar{\ensuremath{t\bar{t}}}
\newcommand\tttt{\ensuremath{t\bar{t}t\bar{t}}}
\newcommand\ttH{\ensuremath{t\bar{t}H}}
\newcommand\ttZ{\ensuremath{t\bar{t}Z}}
\newcommand\ttW{\ensuremath{t\bar{t}W}}
\newcommand\bbbar{\ensuremath{b\bar{b}}}
\newcommand{\met}{\ensuremath{E_{{T}}^{{miss}}}}
\newcommand{\pt}{\ensuremath{p_{T}}}

\newif\ifAddReferences  %% References
\newif\ifAddStatement  %% Statement of research interest
\newif\ifAddInternalTalks  %% References
\newif\ifAddTalks  %% References
\AddReferencesfalse
\AddStatementfalse
\AddInternalTalksfalse
\AddTalkstrue

\AtBeginDocument{\hypersetup{colorlinks,citecolor=blue,linkcolor=blue,urlcolor=blue}}
\AtBeginDocument{\renewcommand\refname{~}}

\name{Javier}{Montejo Berlingen} %many FIXME around the CV, find them
%%%%%%%%%%%%%%%%%%%%%%
% SINCE YOU ARE READING THIS, LET ME REMIND YOU ABOUT THE IDEA OF HAVING A VERTICAL LINE IN THE MIDDLE
% AND THE RESULTS AND POSITIONS ON BOTH SIDES TO GIVE CONTEXT OF THE TIMELINE
%                                                                  CHECK 'CV nuevo layout.key' <----------
%%%%%%%%%%%%%%%%%%%%%%
\title{CERN Staff Physicist}                               % optional, remove / comment the line if not wanted
%\address{CERN 40/5-C11}{1217 Meyrin}{Switzerland}
%\phone[fixed]{+41~786314562}
%\email{jmontejo@cern.ch}                               % optional, remove / comment the line if not wanted


\usepackage{multibib}
\newcites{article,confnote,proceedings}{{Articles},{Conference Notes},{Proceedings}}


%----------------------------------------------------------------------------------
%            content
%----------------------------------------------------------------------------------
\begin{document}
%-----       resume       ---------------------------------------------------------
\makecvtitle
\vspace*{-15mm}

\begin{tikzpicture}[remember picture,overlay,shift={(current page.north east)}]
\node[anchor=north east,xshift=-2cm,yshift=-2.3cm]{\includegraphics[width=5cm]{Javi_foto_curriculum_gimp_crop}};
\end{tikzpicture}

\section{Personal Information}
\cvline{Date of birth}{March 26, 1986}
%\cvline{Place of birth}{Salamanca, Spain}
%\cvline{Sex}{Male}
\cvline{Nationality}{Spanish, German}
\cvline{Email}{jmontejo@cern.ch}
\cvline{Phone}{+41 786314562}

\vspace*{5mm}




\section{Education and Research Positions}
\cventry{2017 - today}{CERN Staff Physicist}{}{}{
\newline{}SUSY R-parity violating and long-lived subgroup convener.
\newline{}Trigger Level-1 calorimeter algorithm and performance coordinator.
\newline{}Trigger menu and performance coordinator. 
\newline{}Member of Physics Coordination group.
\newline{}Member and Scientific Secretary of the Trigger Coordination Group.
\newline{}Supervisor of two CERN fellows.}{}
\cventry{2015 - 2017}{CERN Fellow}{}{}{\newline{}SUSY third-generation subgroup convener.}{}
\cventry{2012 - 2015}{Ph.D.,}{Universitat Aut\'onoma de Barcelona,}{Spain.}{\newline{}
ATLAS thesis award.\newline{}Springer thesis award.}{}
\cventry{2011 - 2012}{M.Sc. in High-Energy Physics,}{Universitat Aut\'onoma de Barcelona,}{Spain.}{}{}
\cventry{2005 - 2011}{B.Sc. in Computer Science,}{Universidad de Salamanca,}{Spain.}{\newline{}
Extraordinary Graduation Award.}{}
\cventry{2005 - 2010}{B.Sc. in Physics,}{Universidad de Salamanca,}{Spain.}{\newline{}Extraordinary Graduation Award.\newline{}National Award for Excellence in Academic Performance.}{}{}

\section{Outreach and mentoring}
\cvitem{}{Beamline for schools (B4LS) 2020, assistant and analysis support.}
\cvitem{}{International masterclass 2016--2018, moderator and virtual visits.}
\cvitem{}{World of work 2018, host and supervisor.}
\cvitem{}{Supervisor of three summer students, 2016--2019}
\cvitem{}{ATLAS guide since 2015.}


\ifAddStatement
\clearpage
%\setlength{\hintscolumnwidth}{0cm}
%\setlength{\maincolumnwidth}{15.6cm}
\section{Statement of research interest}

%%%%%%%%% Intro and hierarchy probelm
\cvitem{}{
After the successful observation of the Higgs boson in Run 1, the increased energy of Run 2 has provided the LHC experiments with an unprecedented dataset to explore the energy frontier. No significant excess has been observed so far, and stringent limits have been set on simplified models.
The motivation for some form of new physics is still strong, but it is obvious that it is not manifested in the vanilla signatures that were the main focus of attention.
The Higgs boson mass has been measured at the electroweak scale, and the corresponding \emph{hierarchy problem} remains an unsolved question that I would like to address.
}

%%%%%%%%% Natural supersymmetry and RPV
\cvitem{}{
Natural supersymmetry remains a prime candidate to solve the hierarchy problem, and in particular R-parity violating (RPV) supersymmetry models can feature naturally light mass spectra with only weak exclusion limits. One of the key ingredients in natural supersymmetry is the presence of light higgsinos. Searches for higgsino production in final states with missing energy are well established at the LHC, and will continue progressing. However, higgsino searches without missing energy, as predicted by RPV, are largely uncovered. My goal is to develop a program to cover the natural region of light higgsinos, including the most challenging decays.}
%Ewk production slepton-mediated > 1 TeV

%%%%%%%%% multi-lepton multi-bjet and LQD
\cvitem{}{
I have developed an analysis targeting higgsino production with RPV decay via a baryon-number-violating coupling $UDD$, exploiting the final state with at least one lepton and many $b$-jets ($\geq 1 \ell, \geq 3$ b-jets).
This search provides the first limits on hadronic higgsino decays since LEP. I would like to extend the analysis into the multi-leptons multi-$b$-jets final state ($\geq 3 \ell, \geq 3$ b-jets), in order to target also effectively decays via the lepton-number-violating coupling $LQD$, which is so far uncovered. 
Supersymmetric models with this coupling have received renewed interest recently given their potential to explain the $R_K$, $R_K^*$, and muon $g-2$ anomalies.
}

%%%%%%%%% ttW+bb measurement
\cvitem{}{
This final state is certainly not new, and searches have already been performed, focusing however on high-mass signals. 
For low-mass signals, the lack of striking distinguishing features requires an excellent control of the background modelling in order to fully capitalise on the possibilities of this channel.
The main background in this final state originates from \ttW\ with additional heavy flavour jets. %, and to a lesser extent \ttbar\ with additional fake or non-prompt leptons. 
Despite the availability of NLO predictions for the \ttW\ process, the modelling uncertainties on this background remain large, and several analyses observe data/MC disagreements above 50\%. Keeping in mind the long-term possibilities of this final state, I consider that a differential measurement of the \ttW\ process, in particular with additional heavy-flavour jets, is a required first step to establish the program.
}

%%%%%%%%% BSM in 3l3b
\cvitem{}{
With backgrounds under control, a search for higgsinos decaying via $LQD$ can be performed. 
%Besides the usual search for excesses in certain regions of phase space, a search for asymmetries in lepton flavours can also target the most motivated couplings, and provide 
It is worth highlighting that searches in multi-lepton and multi-$b$-jets final states are not exclusive to RPV supersymmetry. Other RPC models such as stealth supersymmetry, NMSSM, or longer decay chains within the electroweak superpartners also predict similar final states. %, as well as non-supersymmetric models such as 2HDMs or vector-like quarks. 
A common aspect of these models is the possibility for higgsinos to decay via multiple vector bosons and higgs bosons or additional scalars, depleting the final state from \met. As part of the program, a search for b\=b resonances in this final state would tackle these kind of models. 
%Therefore, light higgsinos with such challenging decays have not been tested yet, and could also be accessed with the multi-lepton multi-$b$-jet final state.
Therefore, the multi-lepton multi-$b$-jet final state can be used to target a large variety of models featuring light higgsinos with challenging decays, that have not been tested so far.
}

%%%%%%%%% long-lived 
\cvitem{}{
In order to achieve an exhaustive coverage of the higgsino landscape, a possibility that should not be neglected is that the higgsinos could be long-lived. 
%The numerical values of the RPV couplings are often considered a free parameter of the theory. 
A variety of models that include RPV terms also predict very small values for the couplings, which lead to displaced decays. Searches for displaced decays, such as displaced vertices, have been conducted at the LHC, however focusing mostly on strong production. 
The selections and reconstruction techniques that target gluinos and R-hadrons above 2 TeV are extremely inefficient in the search for higgsinos at the electroweak scale. %they might be familiar with H 4b -like DVs
%Searches for displaced decay products from low-mass scalars also exist, but rely on the associated production with a vector boson, which is not possible in the case of higgsinos.
In addition, the background levels increase dramatically when attempting to adapt current searches towards vertices with lower masses and lower number of tracks, as required to have acceptance to lower mass scales.
Therefore, I would like to pursue a program for electroweak-scale displaced decays, leading to displaced vertices. 
%And in particular, focusing on processes such as higgsino production, that do not benefit from the possibility of associated production with a vector boson. 
}

%%%%%%%% DV with soft muons
\cvitem{}{
From an experimental point of view, one of the main handles to identify the signals previously discussed was the presence of a high number of $b$-jets. This feature is however seemingly lost when considering displaced decays. One possibility to recover it however, is to exploit B-hadron decays to muons. By requiring a displaced vertex with one or more muons attached, the backgrounds can be massively reduced, enabling a search for displaced vertices with lower masses and lower number of tracks. Other interesting signals that can be accessed lowering the track multiplicity requirements are for example decays to two displaced taus. The low production cross section and additional penalty from the muonic branching ratio of B-hadrons requires a large recorded dataset, but the achieved purity can be excellent, making this a suitable target for the upcoming LHC runs. 
%
}

%%%%%%%% trigger
\cvitem{}{
One of the main challenges of such program is the absence of an effective trigger. The final state contains no significant \met, the low mass scale that is targeted is significantly below the hadronic trigger thresholds, and possible leptons in the decay are displaced, rendering the usual lepton triggers inefficient. Many searches rely on the associated production with a vector boson to trigger. However, this option is not present in the case of higgsino production, as well as other models, and other trigger strategies need to be developed.
Two different approaches can and should be pursued. One on side, triggering directly on the displaced decay products can provide a high efficiency at the cost of some model-dependence. In particular I would like to develop a trigger for displaced vertices with additional displaced leptons in the vertex. The leptons can originate either from the BSM particle, top quarks, or b-quarks in the decay. The lepton provides handles to reduce the rate, and defines a region in the detector where the displaced vertex reconstruction can be run. Limiting therefore the rate and area where the CPU-intensive reconstruction of large-radius tracks has to run.
The second approach to triggering would be to rely on the products of some associated production. The associated production with a vector boson is not possible for a higgsino signal, however the production via vector-boson-fusion can be exploited for this purpose. Triggering on a high-mass dijet system with large separation in $\eta$ would allow to reconstruct and trigger also on displaced vertices without leptons, opening up possibilities to target also low-mass signals without leptons or heavy-flavour quarks.
%HGTD?
}

%\cvitem{}{
%The ATLAS measurement program has been extremely successful, and processes with cross-sections as low as $\mathcal{O}(1)$ fb have been measured. However it is important to keep in mind that so far less than 5\% of the target of 3000 $\text{fb}^{-1}$ has been recorded. The large increase in luminosity will benefit especially processes with low cross-sections, and final states with high purity but low branching ratios, such as multi-lepton final states. In addition, it will also transform the treatment of objects where the choice of working point means a trade-off between purity and statistics. Examples of such are identification and isolation of leptons, or tagging of jets originating from $b$-quarks ($b$-jets). 
%For the aforementioned reasons, I consider that a program of measurements in final states with multiple leptons and $b$-jets will become extremely valuable during the next years, and up to the high-luminosity phase of LHC (HL-LHC). 
%}

%%%%%%%Conclusion
\cvitem{}{
As discussed, I consider that a program of searches for higgsinos with challenging decays is needed to cover as exhaustively as possible one of the key predictions of natural supersymmetry. Searches in final states with multiple leptons and $b$-jets have a large potential to explore such challenging decays, although precise measurements of the main background will be needed to fully capitalise on the possibilities of this final state. Extending such searches into the regime of long-lived decays is difficult but certainly possible. Dedicated triggers will need to be developed to effectively record such events. In both prompt and displaced searches, the lepton and $b$-jet multiplicities and identification criteria offer a continuous trade-off between purity and available statistics. Therefore proving the flexibility to fully exploit the fast increase in recorded dataset during the next years, and up to the high-luminosity phase of LHC.}
	
%%%%%%%Conclusion
\cvitem{}{
\color{red} I'm a bit concerned about the balance between what I want to do and why I want to do it. I feel like I might be overemphasizing the discussion about models that can be targeted an so on. Also the interplay with existing searches and why those can't be used. I'm obviously not a hardware person, but I feel like I could highlight a bit more the impact of the planned upgrades.}
%\setlength{\hintscolumnwidth}{2.05cm}
\fi

\clearpage
\section{Research Experience}

%%%%%%%%%%%%%%%% ttH(bb)
\cvitem{{Run~1: t\=tH(b\=b)}}{
At the beginning of my PhD studies, the Higgs boson had recently been discovered, but most of its properties remained to be measured, in particular the couplings to quarks.
I decided to concentrate on the associated production of a Higgs boson and a top-quark pair (\ttH), given the important role of the top Yukawa coupling which, due to the loop corrections to the Higgs mass, is the main contributor to the hierarchy problem.
%is the only ``natural'' Yukawa coupling, with $\lambda_\text{t}\approx 1$.
%Measuring the top Yukawa via the \ttH\ production cross section also provides a handle to identify possible BSM contributions in the dominant Higgs production mode, which is the loop-induced gluon-gluon fusion process, and also the very sensitive decay to two photons.
%In addition, due to the loop corrections to the Higgs mass, the top Yukawa coupling is the main contributor to the hierarchy problem.
}
\cvitem{}{
In order to compensate for the low \ttH\ production cross section, I decided to target the decay of the Higgs boson to \bbbar\ and semileptonic \ttbar\ decays.
I focused on the development of the selection, fitting strategy and background estimation, in particular the challenging \ttbb\ background modelling and associated systematic uncertainties.
A collaboration with theorists was started to model the \ttbb\ process at NLO for the first time. I worked on the integration of this improved modelling in the \ttbb\ subset within the inclusive \ttbar+jets sample, via a multi-dimensional reweighting.
}
\cvitem{}{
The results of the \ttH\ analysis were summarised in publications at 7 TeV and 8 TeV. The analysis was included in the ATLAS Higgs coupling combination, and further combined with CMS, leading to the first evidence of \ttH\ production.
}

%%%%%%%%%%%%%%%%% BSM ttbb
\cvitem{{Run~1: BSM t\=tb\=b}}{
The confirmation of the SM-like nature of the top Yukawa coupling (at least to first order), implied also the corroboration of large loop corrections to the Higgs mass, leading to the hierarchy problem. I therefore decided to explore BSM signatures in the \ttbb\ final state, that could provide solutions to the hierarchy problem via fermionic top partners (vector-like quarks, VLQ) or bosonic top partners (stops) cancelling the contribution from top quarks. Both searches involved a full redesign of the previous \ttH(bb) analysis, tailoring the selection and fitted variables to the considered signals. 
}
\cvitem{}{
The search for a VLQ decaying to a Higgs boson and a top quark ($T\rightarrow H t$) required an improved understanding of the \ttbb\ background in the high-energy and boosted regime. The analysis included, for the first time, combinations with other VLQ searches and set limits on the VLQ mass regardless of its decay branching ratios.}
\cvitem{}{
The second search targeted supersymmetric models where the mass difference between the lightest stop and neutralino is close to the top mass, where existing searches had no sensitivity. The search exploited the production of the heavier stop and the subsequent decay to the lightest stop and Higgs boson ($\tilde{t}_2 \rightarrow \tilde{t}_1 H$). The presence of additional missing transverse energy (\met) can be required to suppress backgrounds, but this increases the importance of previously-subdominant backgrounds such as dileptonic \ttbb\ with one lost lepton.
The results of the search were combined with existing searches in complementary decay channels ($\tilde{t}_2 \rightarrow \tilde{t}_1 Z$, and $\tilde{t}_2 \rightarrow t \tilde{\chi}^0_1$), to provide limits on the stop mass regardless of its decay modes.
}

\cvitem{Run~1: Tile calorimeter}{During my PhD studies I  also worked on the characterisation and calibration of the timing performance of the Tile calorimeter using collision data. I identified multiple detector and geometrical effects that were previously not accounted for, and introduced a set of selection cuts and corrections that improved the resolution of the time measurement by up to 20\%.}
\cvitem{}{
I also gained some hardware experience by contributing to maintenance and repairs of modules during the LS1 phase.
Building on my work on the calorimeter performance, I contributed as Data Quality Validator and Data Quality Leader for the Tile calorimeter.
}

\cvitem{{Run~2: RPC SUSY}}{
The large increase in energy and luminosity at the start of Run~2 made it the perfect moment to embark on BSM searches, which would benefit the most from the increased centre-of-mass energy. 
As one of the prime candidates to address the hierarchy problem, I focused on searches for supersymmetry, in particular I joined the search for stops in the single-lepton final state.
%My focus was on the challenging regions of phase-space where top squarks with masses well below 1 TeV were not excluded. 
}
\cvitem{}{
I redesigned the background estimation techniques used in the search in order to reduce the reliance on MC simulation, and developed new selections to further suppress backgrounds.
The optimised selection increased the relevance of previously subdominant backgrounds with large uncertainties, such as $\ttZ(\nu\bar{\nu})$ or single top, for which I developed dedicated control regions.
I also contributed to the design of new signal benchmarks, moving away from simplified models in favour of more complex scenarios inspired by feasible and promising pMSSM benchmarks.
One such example is the production of stop pairs with decays to higgsinos, featuring small mass splittings, and leading to a final state with soft leptons ($\pt \lesssim 5$ GeV) which is experimentally challenging.
The strong sensitivity and flexibility of the analysis led to a series of publications with integrated luminosities of 3.2~fb$^{-1}$, 14 fb$^{-1}$ and 36 fb$^{-1}$, with increasingly comprehensive coverage of stop production within natural SUSY scenarios.
}
\cvitem{}{
During this period I was appointed as convener of the third-generation SUSY subgroup, where I managed to bring the programme of searches with 36~fb$^{-1}$ of data to a timely completion, and lead the preparation of analyses for the full Run 2 results. I initiated a restructuring of the analysis groups to merge searches for third-generation squarks and searches for dark matter with associated heavy-flavour quarks, leading to stronger and better organised teams that delivered impressive results on both models.
}

\cvitem{{Run~2: RPV SUSY}}{
Although supersymmetry is a very compelling extension of the SM, no hint for vanilla SUSY has been observed. This motivates the extension of the searches to less traditional final states. I have developed a completely new search for RPV SUSY in final states with no \met, at least one lepton, and a high number of jets (up to 15 jets and 4 $b$-jets). The absence of \met\ in the final state is a feature that could cause supersymmetric particles to remain unobserved due to the large requirements on \met\ placed by most searches. The background modelling at very high jet multiplicities is extremely challenging, and I developed new data-driven methods to estimate the background. The first iterations of the search focused on strong production of gluinos and stops, and set stringent exclusion limits on models that were previously completely uncovered. Due to its design as a model-independent search, I also contributed to its reinterpretation for dark matter models yielding four-top production.}
\cvitem{}{
The analysis with full Run 2 data had as its target benchmark higgsino production. In order to reach sensitivity to such low cross sections, I introduced machine-learning techniques, where custom modifications of the loss function allow the discriminant to be invariant with respect to the number of $b$-jets. This property is exploited to extend the data-driven background estimation to the shape of the discriminant. The sensitivity of the analysis is strongly increased with no additional reliance on MC. 
The search reaches sensitivity to higgsino production with RPV decay to quarks, and is the first analysis to surpass the limits from LEP.
%Highlighted in the CERN courier https://cerncourier.com/wp-content/uploads/2021/04/CERNCourier2021MayJun-digitaledition.pdf
}
\cvitem{}{
I am currently the convener of the RPV and long-lived SUSY subgroup. During my term I have proposed several analyses in previously-unexplored final states, which are now being developed. I also initiated a taskforce to explore the interplay of RPC analyses, prompt RPV analyses, and long-lived searches when considering models where the lifetime of the lightest supersymmetric particle could vary from prompt, long-lived, or collider-stable. In this context, I coordinated the work in collaboration with performance groups to develop recommendations for the uncertainties on lepton, jet, $b$-jet and \met\ calibrations, when reconstructing non-prompt decays. The results showed explicitly, for the first time, the good coverage provided across the lifetime range of the different analyses, and also highlighted some sensitivity gaps where dedicated searches would be needed.
}

\cvitem{{Run~2: BSM Higgs}}{
So far the properties of the discovered Higgs boson are in good agreement with the SM predictions. However, additional possible decays are still possible, with branching ratios of up to $\mathcal{O}(10\%)$, as well as additional bosons from extended Higgs sectors. I have contributed to the design and optimisation of an analysis targeting the decay of the Higgs boson to two light pseudo-scalars, which in turn decay to pairs of b-quarks. 
%The pseudo-scalar decays are reconstructed as single jets with two B-hadrons inside. 
The main background originates from gluon splitting to b-quarks with small angular separation, a regime which suffers from large modelling uncertainties. In order to overcome this I devised a selection and defined analysis regions with very low sensitivity to modelling uncertainties from such backgrounds.
}
\cvitem{}{
In addition, I am currently working on an analysis targeting models where the heavy Higgs sector can exhibit flavour-changing neutral currents. This kind of models feature a rich phenomenology, and can produce yet unexplored signatures such as three-top production.
}

\cvitem{Run~2: Trigger}{
I  contributed to the ATLAS trigger menu group throughout Run~2, initially as trigger menu expert and menu on-call, and during 2018--2019 as trigger menu and signature coordinator. As such, I was in charge of coordinating the work of all the signature, detector and physics trigger groups, which encompassed more than a hundred members. In this role I was part of the Physics Coordination group, given the menu's critical relevance to the ATLAS physics programme. 
During this time I was in charge of defining the menus for the intense period of special runs in 2018, including the low-$\mu$, high-$\beta^*$, and the Heavy-Ion run. I was also responsible for the documentation of the trigger menu evolution in dedicated public notes.
After the Run 2 data-taking period, I defined the ATLAS physics menu for Run~3. I successfully argued for a strongly increased recording rate for Run 3, which was endorsed by the experiment. Besides the menu composition, I contributed to dedicated studies in order to ensure that the trigger menu being designed would respect all the hardware constraints of the upgraded systems.
}
\cvitem{}{
%\cvitem{Run~3: Trigger}{
As part of my effort to improve the Run~3 triggers, I am currently Level-1 calorimeter algorithm and trigger performance coordinator.  I am leading the development and optimisation of reconstruction and identification algorithms that can maximise the performance gains from the upgraded calorimeter trigger electronics, exploiting the capabilities of the new system to its full potential.
}

\clearpage
%\section{List of publications}
\cvitem{}{Highlighted in bold are the three publications that I attach as the most significant ones.}

\cvitem{Run~1: ttH(bb)}{
I was one of the main analysers in the first ATLAS analysis targeting the \ttH\ production mode, contributing to the development of the selection, fitting strategy and background estimation. In particular the challenging \ttbb\ background modelling and associated systematic uncertainties, where a collaboration with theorists was started. I studied the \ttbb\ modelling for the first time at NLO, and integrated the dedicated calculation in the inclusive \ttbar+jets sample via a multi-dimensional reweighting.
\begin{itemize}
\item Search for the Standard Model Higgs boson produced in association with top quarks in proton-proton collisions at $\sqrt{s}=7$ TeV using the ATLAS detector~\cite{ATLAS-CONF-2012-135}.
\item \textbf{Search for the Standard Model Higgs boson produced in association with top quarks and decaying into $b\bar{b}$ in pp collisions at $\sqrt{s}=8$ TeV with the ATLAS detector}~\cite{HIGG-2013-27}. 
\item Measurements of the Higgs boson production and decay rates and coupling strengths using pp  collision data at $\sqrt{s}=7$ and 8 TeV in the ATLAS experiment~\cite{HIGG-2014-06}.
\item Measurements of the Higgs boson production and decay rates and constraints on its couplings from a combined ATLAS and CMS analysis of the LHC pp collision data at $\sqrt{s}=7$ and 8 TeV~\cite{HIGG-2015-07}.
\end{itemize}
}

\cvitem{Run~1: BSM ttbb}{
I developed searches for fermionic and bosonic top partners in the \ttbb\ final states. Both searches involved a full redesign of the previous \ttH(bb) analysis, tailoring the selection and fitted variables to the considered signals. For both analyses I was the main (and only) analyser, together with my supervisor.
\begin{itemize}
\item Search for production of vector-like quark pairs and of four top quarks in the lepton-plus-jets final state in pp collisions at $\sqrt{s}=8$ TeV with the ATLAS detector~\cite{EXOT-2013-18}.
\item ATLAS Run 1 searches for direct pair production of third-generation squarks at the Large Hadron Collider~\cite{SUSY-2014-07}.
\end{itemize}
}

\cvitem{Tile calorimeter}{
I worked in the characterisation of the Tile calorimeter timing performance and calibration with muons from collision events.
\begin{itemize}
\item Operation and performance of the ATLAS Tile Calorimeter in Run 1~\cite{TCAL-2017-01}.
\end{itemize}
}

\cvitem{Run~2: RPC SUSY}{
I was analysis contact in the search for top squarks in the single-lepton final state, leading a group of around 15 people. Redesigned the background estimation methods to reduce the reliance on MC simulation, and developed new selections to further suppress backgrounds. I also defined new signal benchmarks, in order explore more comprehensively challenging models with low stop masses.
\begin{itemize}
\item Search for top squarks in final states with one isolated lepton, jets, and missing transverse momentum in $\sqrt{s}=13$ TeV pp collisions with the ATLAS detector~\cite{SUSY-2015-02}.
\item Search for top squarks in final states with one isolated lepton, jets, and missing transverse momentum in $\sqrt{s}=13$ TeV pp collisions with the ATLAS detector~\cite{ATLAS-CONF-2016-050}. 
\item \textbf{Search for top-squark pair production in final states with one lepton, jets, and missing transverse momentum using 36 fb$^{-1}$ of $\sqrt{s}=13$ TeV pp collision data with the ATLAS detector}~\cite{SUSY-2016-16}.
\end{itemize}
}

\cvitem{Run~2: RPV SUSY}{
I co-designed with another CERN fellow a fully new analysis targeting the final state of a lepton plus many jets (up to 15 jets and 4 $b$-jets), which was previously uncovered. The main challenge was the background estimation at these extreme multiplicities for which we developed fully new data-driven methods. I am also paper editor for the full Run 2 paper. Due to its wide applicability I also contributed to its reinterpretation in four-top models, and SUSY models with displaced decays, where I was also CONF editor.
\begin{itemize}
\item Search for new phenomena in a lepton plus high jet multiplicity final state with the ATLAS experiment using $\sqrt{s}=13$ TeV proton-proton collision data~\cite{SUSY-2016-11}.
\item \textbf{Search for R-parity violating supersymmetry in a leptons plus high jet multiplicity final state with the ATLAS experiment using 139 fb$^{-1}$ of $\sqrt{s}=13$ TeV proton--proton collision data}~\cite{RPV1L} (in approval).
\item Constraints on mediator-based dark matter and scalar dark energy models using $\sqrt{s}=13$ TeV pp collision data collected by the ATLAS detector~\cite{EXOT-2017-32}.
\item Reinterpretation of searches for supersymmetry in models with variable R-parity-violating coupling strength and long-lived R-hadrons~\cite{ATLAS-CONF-2018-003}.
\end{itemize}
}


\cvitem{Trigger}{
I was editor of the 2017 trigger menu PUB note, and coordinator and supervisor of the 2018 trigger menu PUB note. I also was in charge of the design of the Run 3 trigger menu.
\begin{itemize}
\item Trigger Menu in 2017~\cite{ATL-DAQ-PUB-2019-001}.
\item Trigger Menu in 2018~\cite{ATL-DAQ-PUB-2018-002}.
\item Run 3 trigger menu design~\cite{Run3menu}.
\end{itemize}
}


\section{List of publications}
\cvitem{}{A list of the 10 most important publications where I have been the main contributor is given below. The full list of publications can be found in the CV.}

\vspace{0.5cm}
\cvitem{\textbf{t\=tH(b\=b) and BSM t\=tb\=b}}{
Lead analyser for the t\=tH(b\=b), VLQ ($T \rightarrow Ht$) and stop ($\tilde{t}_2 \rightarrow \tilde{t}_1 H$) analyses. My t\=tH(b\=b) result was the single most sensitive analysis to t\=tH in Run 1. The VLQ analysis was the strongest one among the analyses entering the combination, and provided world-best limits at the time.
}
\begin{thebibliography}{10}
%\small
\makeatletter
\renewcommand\@biblabel[1]{}
\makeatother
\expandafter\ifx\csname url\endcsname\relax
  \def\url#1{\texttt{#1}}\fi
\expandafter\ifx\csname urlprefix\endcsname\relax\def\urlprefix{URL }\fi
\expandafter\ifx\csname href\endcsname\relax
  \def\href#1#2{#2} \def\path#1{#1}\fi

\bibitem{HIGG-2013-27}
{ATLAS Collaboration}, {\em Search for the Standard Model Higgs boson produced in
  association with top quarks and decaying into \(b\bar{b}\) in \(pp\)
  collisions at \(\sqrt{s} = 8\,\text{TeV}\) with the ATLAS detector}, 
  \coloredLink{http://dx.doi.org/10.1140/epjc/s10052-015-3543-1}
  {Eur. Phys. J. C 75 (2015) 349}.


\bibitem{EXOT-2013-18}
{ATLAS Collaboration}, {\em Search for production of vector-like quark pairs and of
  four top quarks in the lepton-plus-jets final state in \(pp\) collisions at
  \(\sqrt{s} = 8\,\text{TeV}\) with the ATLAS detector}, 
  \coloredLink{http://dx.doi.org/10.1007/JHEP08(2015)105}
  {JHEP 08 (2015) 105}.

\bibitem{SUSY-2014-07}
{ATLAS Collaboration}, {\em ATLAS Run 1 searches for direct pair production of
  third-generation squarks at the Large Hadron Collider}, 
  \coloredLink{http://dx.doi.org/10.1140/epjc/s10052-015-3726-9}
  {Eur. Phys. J. C 75 (2015) 510}.
\end{thebibliography}


\vspace{0.5cm}
\cvitem{\textbf{Tile calorimeter}}{
Contributor to the characterisation of the Tile calorimeter timing performance and calibration, documented as part of the Tile Calorimeter paper.
}
\begin{thebibliography}{10}
%\small
\makeatletter
%\addtocounter{\@listctr}{3}
\renewcommand\@biblabel[1]{}
\makeatother
\expandafter\ifx\csname url\endcsname\relax
  \def\url#1{\texttt{#1}}\fi
\expandafter\ifx\csname urlprefix\endcsname\relax\def\urlprefix{URL }\fi
\expandafter\ifx\csname href\endcsname\relax
  \def\href#1#2{#2} \def\path#1{#1}\fi

\bibitem{TCAL-2017-01}
{ATLAS Collaboration}, {\em Operation and performance of the ATLAS Tile Calorimeter
  in Run 1},
  \coloredLink{http://dx.doi.org/10.1140/epjc/s10052-018-6374-z}
  {Eur. Phys. J. C 78 (2018) 987}.
  
\end{thebibliography}


\vspace{0.5cm}
\cvitem{\textbf{RPC SUSY}}{
Main analyser, analysis contact and paper editor. 
The signal benchmarks, background estimation methods, and analysis strategy that I implemented in the analysis have continued being used for the publication with full Run 2 data, highlighting the robust work that went into this set of searches.
}
\begin{thebibliography}{10}
%\small
\makeatletter
%\addtocounter{\@listctr}{4}
\renewcommand\@biblabel[1]{}
\makeatother
\expandafter\ifx\csname url\endcsname\relax
  \def\url#1{\texttt{#1}}\fi
\expandafter\ifx\csname urlprefix\endcsname\relax\def\urlprefix{URL }\fi
\expandafter\ifx\csname href\endcsname\relax
  \def\href#1#2{#2} \def\path#1{#1}\fi
  
\bibitem{SUSY-2015-02}
{ATLAS Collaboration}, {\em Search for top squarks in final states with one
  isolated lepton, jets, and missing transverse momentum in \(\sqrt{s} =
  13\,\text{TeV}\) \(pp\) collisions with the ATLAS detector},
  \coloredLink{http://dx.doi.org/10.1103/PhysRevD.94.052009}
  {Phys. Rev. D 94 (2016) 052009}.


\bibitem{SUSY-2016-16}
{ATLAS Collaboration}, {\em Search for top-squark pair production in final states
  with one lepton, jets, and missing transverse momentum using
  \(36\,\text{fb}^{-1}\) of \(\sqrt{s} = 13\,\text{TeV}\) \(pp\) collision data
  with the ATLAS detector}, 
  \coloredLink{http://dx.doi.org/10.1007/JHEP06(2018)108}
  {JHEP 06 (2018) 108}.
  
\end{thebibliography}

\vspace{0.5cm}
\cvitem{\textbf{RPV and long-lived SUSY}}{
Main analyser, analysis contact and paper editor. I developed from scratch fully new data-driven techniques to estimate backgrounds in previously-uncovered final states. The last result is the first LHC search to surpass the limits on higgsinos with RPV decays set by LEP. In a reinterpretation of existing results, I also demonstrated explicitly for the first time the good sensitivity of prompt searches to signals with long lifetime. 
}
\begin{thebibliography}{10}
%\small
\makeatletter
%\addtocounter{\@listctr}{6}
\renewcommand\@biblabel[1]{}
\makeatother
\expandafter\ifx\csname url\endcsname\relax
  \def\url#1{\texttt{#1}}\fi
\expandafter\ifx\csname urlprefix\endcsname\relax\def\urlprefix{URL }\fi
\expandafter\ifx\csname href\endcsname\relax
  \def\href#1#2{#2} \def\path#1{#1}\fi

\bibitem{SUSY-2016-11}
{ATLAS Collaboration}, {\em Search for new phenomena in a lepton plus high jet
  multiplicity final state with the ATLAS experiment using \(\sqrt{s} =
  13\,\text{TeV}\) proton--proton collision data}, 
  \coloredLink{http://dx.doi.org/10.1007/JHEP09(2017)088}
  {JHEP 09 (2017) 088}.
  
  
\bibitem{ATLAS-CONF-2018-003}
{ATLAS Collaboration}, {\em Reinterpretation of searches for
  supersymmetry in models with variable \(R\)-parity-violating coupling
  strength and long-lived \(R\)-hadrons}, 
  \coloredLink{https://atlas.web.cern.ch/Atlas/GROUPS/PHYSICS/CONFNOTES/ATLAS-CONF-2018-003/}{ATLAS-CONF-2018-003}.  

\bibitem{SUSY-2019-04}
{ATLAS Collaboration}, {\em {Search for R-parity-violating supersymmetry in a
  final state containing leptons and many jets with the ATLAS experiment using
  \(\sqrt{s} = 13\,\text{TeV}\) proton--proton collision data}\/},
  \coloredLink{http://dx.doi.org/10.1140/epjc/s10052-021-09761-x}{Eur. Phys. J. C
  {81} (2021)  1023}.

\end{thebibliography}

\vspace{0.5cm}
\cvitem{\textbf{Trigger}}{
Editor of the trigger menu public notes documenting the content and evolution of the physics menu, including the special runs.
}
\begin{thebibliography}{10}
%\small
\makeatletter
%\addtocounter{\@listctr}{9}
\renewcommand\@biblabel[1]{}
\makeatother
\expandafter\ifx\csname url\endcsname\relax
  \def\url#1{\texttt{#1}}\fi
\expandafter\ifx\csname urlprefix\endcsname\relax\def\urlprefix{URL }\fi
\expandafter\ifx\csname href\endcsname\relax
  \def\href#1#2{#2} \def\path#1{#1}\fi

\bibitem{ATL-DAQ-PUB-2019-001}
{ATLAS Collaboration}, {\em Trigger Menu
  in 2018}, \coloredLink{https://atlas.web.cern.ch/Atlas/GROUPS/PHYSICS/PUBNOTES/ATL-DAQ-PUB-2019-001/}{ATL-DAQ-PUB-2019-001}.  
\end{thebibliography}




\ifAddTalks
\clearpage
\ifAddInternalTalks
\section{Conferences, schools and workshops}
\else
\section{Conference talks}
\fi
\cventry{Jun 2021}{Large Hadron Collider Physics (LHCP 2021),}{\sout{Paris},}{Virtual}{\newline{}
BSM Higgs decays at ATLAS+CMS
}{}{}
\cventry{Jul 2020}{40th International Conference on High Energy Physics (ICHEP 2020),}{\sout{Prague},}{Virtual}{\newline{}
Triggering in the ATLAS experiment
    [\coloredLink{https://cds.cern.ch/record/2728145}{ATL-DAQ-SLIDE-2020-320}]
    [\coloredLink{https://cds.cern.ch/record/2742661}{Proceedings}]}{}{}
\cventry{May 2019}{27th International Conference on Supersymmetry and Unification of Fundamental Interactions (SUSY 2019),}{Corpus Christi,}{Texas}{\newline{}
Searches for supersymmetry in R-parity violating signatures at the LHC.
    [\coloredLink{https://cds.cern.ch/record/2676781}{ATL-PHYS-SLIDE-2019-242}]
}{}{}
\ifAddInternalTalks
\cventry{May 2019}{Trigger workshop,}{Elba,}{Italy.}{\newline{}
Run-3 baseline menu.}{}{} %https://indico.cern.ch/event/772409/
\cventry{Sep 2018}{TDAQ week,}{Krakow,}{Poland.}{\newline{}
Menu considerations for Run 3.}{}{} %https://indico.cern.ch/event/730816
    \fi
\cventry{Jul 2018}{23rd International Conference on Computing in High Energy and Nuclear Physics (CHEP 2018),}{Sofia,}{Bulgaria.}{\newline{}
The ATLAS Trigger Menu design for higher luminosities in Run 2. [\coloredLink{https://cds.cern.ch/record/2631630}{ATL-DAQ-SLIDE-2018-500}][\coloredLink{https://cds.cern.ch/record/2645347}{Proceedings}]
}{}{} 
\cventry{May 2018}{(Re)interpreting the results of new physics searches at the LHC (ReINPS2018),}{CERN,}{Geneva}{\newline{}
Sensitivity of prompt searches to long-lives particles.
    [\coloredLink{https://cds.cern.ch/record/2319790}{ATL-PHYS-SLIDE-2018-282}]
}{}{}
\ifAddInternalTalks
\cventry{Apr 2018}{Four-tops workshop,}{CERN,}{Geneva}{\newline{}
Data-driven methods in lepton+jets searches to evaluate the ttbar+jets and W+jets backgrounds.}{}{}
    \fi
\cventry{May 2017}{Large Hadron Collider Physics (LHCP),}{Shangai,}{China.}{\newline{}
Unconventional signatures and RPV supersymmetry. Plenary. [\coloredLink{https://cds.cern.ch/record/2266308}{ATL-PHYS-SLIDE-2017-310}]
 \newline{}
The ATLAS Run-2 Trigger Menu for higher luminosities: Design, Performance and Operational Aspects. [\coloredLink{https://cds.cern.ch/record/2265272}{ATL-DAQ-SLIDE-2017-255}] }{}{}
\ifAddInternalTalks
\cventry{May 2017}{ATLAS Exotics \& SUSY workshop,}{Bucharest,}{Romania.}{\newline{}
RPV searches in ATLAS.}{}{}
\cventry{Feb 2017}{ATLAS trigger workshop,}{University of Geneva,}{Switzerland.}{\newline{}
Trigger rates and processing time. Pileup dependency.}{}{}
\cventry{Oct 2016}{ttH workshop,}{CERN,}{Geneva.}{\newline{}
ttH to invisible.}{}{}
\cventry{Sep 2016}{TDAQ week,}{Barcelona,}{Spain.}{\newline{}
Trigger menu rates and CPU projections in 2017.}{}{}
\cventry{Apr 2016}{Supersymmetry workshop,}{Sussex,}{UK.}{\newline{}
RPV analyses in ATLAS.}{}{}
    \fi
\cventry{Mar 2015}{XXIX Rencontres de Physique de la Vall\'{e}e d'Aoste,}{La Thuile,}{Italy.}{\newline{}
Search for the Higgs boson in the ttH production channel using the ATLAS detector. 
    %\newline
     [\coloredLink{https://cds.cern.ch/record/2002318}{ATL-PHYS-SLIDE-2015-073}]
    }{}{}
\ifAddInternalTalks
\cventry{Nov 2014}{ttH workshop,}{CERN,}{Geneva.}{\newline{}
Sensitivity projections for ttH.}{}{}
\cventry{Jun 2014}{European School of High-Energy Physics,}{Garderen,}{Holland.}{}{}
\cventry{Sep 2013}{ATLAS Higgs to \bbbar\ workshop,}{Marseille,}{France.}{\newline{}
ttbar modelling for VH and ttH.}{}{}
    \fi
\cventry{Aug 2013}{21st International Conference on Supersymmetry and Unification of Fundamental Interactions (SUSY 2013),}{Trieste,}{Italy.}{\newline{}
Top quark production in the ATLAS detector of the LHC. \newline
     [\coloredLink{https://cds.cern.ch/record/1596998}{ATL-PHYS-SLIDE-2013-509}]
}{}{}
\ifAddInternalTalks
\cventry{Jun 2013}{ttH workshop,}{CERN,}{Geneva.}{\newline{}
tt+jets/bb modelling and systematics in ttH(bb) analyses.}{}{}
    \fi
\cventry{May 2013}{Large Hadron Collider Physics (LHCP 2013),}{Barcelona,}{Spain.}{\newline{}
Search for the Standard Model Higgs boson produced in association with top quarks and decaying to $b\bar{b}$ at $\sqrt{s} = 7~TeV$. \newline
	[\coloredLink{https://cds.cern.ch/record/1551958}{ATL-PHYS-SLIDE-2013-298}]% \newline{}
   [\coloredLink{http://www.epj-conferences.org/articles/epjconf/abs/2013/21/epjconf_lhcp2013_20028/epjconf_lhcp2013_20028.html}{Proceedings}]}{}{}
\ifAddInternalTalks
\cventry{Dec 2012}{ATLAS Higgs to \bbbar\ workshop,}{Saint Genis,}{France.}{\newline{}Talk: Statistical procedures and limit setting in Higgs to \bbbar.}{}{}
\cventry{Jul 2012}{Hadron Collider Physics School (HASCO 2012),}{G\"ottingen,}{Germany.}{}{}
\fi
\cventry{Mar 2012}{LHC commissioning,}{CERN,}{Geneva.}{\newline{}
ATLAS detector performance in 2011: calorimeters. \newline
[\coloredLink{https://cds.cern.ch/record/1445591}{ATL-PHYS-SLIDE-2012-154}]}{}{}
\ifAddInternalTalks
\cventry{Sep 2011}{4th International Workshop on Top Quark Physics (TOP 2011),}{Sant Feliu de Guixols,}{Spain.}{}{}
\fi
\fi


\ifAddReferences
\clearpage
\section{References}

\cventry{}{ \textbf{Andreas Hoecker} }{}{\newline
European Organization for Nuclear Research (CERN) \newline
+41 22 76 74787\newline
andreas.hoecker@cern.ch\newline
}{}{}

\cventry{}{ \textbf{Marumi Kado} }{}{\newline
Laboratoire de l'Accelerateur Lineaire (LAL)\newline
+41 22 76 71143\newline
kado@lal.in2p3.fr\newline
}{}{}

\cventry{}{ \textbf{Aurelio Juste Rozas} }{}{\newline
Institut de Fisica d'Altes Energies (IFAE)\newline
+34 93 581 4249\newline
juste@ifae.es\newline
}{}{}

\cventry{}{ \textbf{Brian Petersen} }{}{\newline
European Organization for Nuclear Research (CERN) \newline
+41 22 76 71199\newline
brian.petersen@cern.ch\newline
}{}{}

\cventry{}{ \textbf{Joerg Stelzer} }{}{\newline
University of Pittsburgh\newline
+41 22 76 78979\newline
stelzer@cern.ch\newline
}{}{}

\fi

\end{document}


%% end of file `template.tex'.
