
\documentclass[11pt,a4paper,sans]{moderncv}        % possible options include font size ('10pt', '11pt' and '12pt'), paper size ('a4paper', 'letterpaper', 'a5paper', 'legalpaper', 'executivepaper' and 'landscape') and font family ('sans' and 'roman')

% moderncv themes
\moderncvstyle{classic}                            % style options are 'casual' (default), 'classic', 'oldstyle' and 'banking'
\moderncvcolor{blue}                               % color options 'blue' (default), 'orange', 'green', 'red', 'purple', 'grey' and 'black'
%\renewcommand{\familydefault}{\sfdefault}         % to set the default font; use '\sfdefault' for the default sans serif font, '\rmdefault' for the default roman one, or any tex font name
%\nopagenumbers{}                                  % uncomment to suppress automatic page numbering for CVs longer than one page

% adjust the page margins
\usepackage[scale=0.73]{geometry}
\setlength{\hintscolumnwidth}{0.0cm}                % if you want to change the width of the column with the dates
%\setlength{\makecvtitlenamewidth}{10cm}           % for the 'classic' style, if you want to force the width allocated to your name and avoid line breaks. be careful though, the length is normally calculated to avoid any overlap with your personal info; use this at your own typographical risks...

\usepackage{color}
\newcommand{\coloredLink}[2]{\textcolor{blue}{\href{#1}{#2}}}

\newcommand\ttbb{\ensuremath{t\bar{t}b\bar{b}}}
\newcommand\ttbar{\ensuremath{t\bar{t}}}
\newcommand\tttt{\ensuremath{t\bar{t}t\bar{t}}}
\newcommand\ttH{\ensuremath{t\bar{t}H}}
\newcommand\ttZ{\ensuremath{t\bar{t}Z}}
\newcommand\ttW{\ensuremath{t\bar{t}W}}
\newcommand\bbbar{\ensuremath{b\bar{b}}}
\newcommand{\met}{\ensuremath{E_{{T}}^{{miss}}}}
\newcommand{\pt}{\ensuremath{p_{T}}}

\newif\ifAddReferences  %% References
\newif\ifAddStatement  %% Statement of research interest
\AddReferencestrue
\AddStatementtrue

\definecolor{moderncvblue}{rgb}{0.22,0.45,0.70}% light blue

% personal data
\name{\color{moderncvblue}Javier}{Montejo Berlingen}
%\title{title}                               % optional, remove / comment the line if not wanted
%\address{CERN 40/5-C11}{1217 Meyrin}{Switzerland}% optional, remove / comment the line if not wanted; the "postcode city" and "country" arguments can be omitted or provided empty
%\phone[fixed]{+41~78~631~45~62}
\phone[fixed]{+41~786314562}
%\phone[fax]{+3~(456)~789~012}
\email{jmontejo@cern.ch}                               % optional, remove / comment the line if not wanted
%\homepage{www.johndoe.com}                         % optional, remove / comment the line if not wanted
%\social[linkedin]{francesco-rubbo}                        % optional, remove / comment the line if not wanted
%\social[twitter]{jdoe}                             % optional, remove / comment the line if not wanted
%\social[github]{francescorubbo}                              % optional, remove / comment the line if not wanted
%\extrainfo{additional information}                 % optional, remove / comment the line if not wanted
%\photo[64pt][0.4pt]{picture}                       % optional, remove / comment the line if not wanted; '64pt' is the height the picture must be resized to, 0.4pt is the thickness of the frame around it (put it to 0pt for no frame) and 'picture' is the name of the picture file
%\quote{some quote}                                 % optional, remove / comment the line if not wanted

% to show numerical labels in the bibliography (default is to show no labels); only useful if you make citations in your resume
%\makeatletter
%\renewcommand*{\bibliographyitemlabel}{\@biblabel{\arabic{enumiv}}}
%\makeatother
%\renewcommand*{\bibliographyitemlabel}{[\arabic{enumiv}]}% CONSIDER REPLACING THE ABOVE BY THIS

\usepackage{multibib}
\newcites{article,confnote,proceedings}{{Articles},{Conference Notes},{Proceedings}}
%----------------------------------------------------------------------------------
%            content
%----------------------------------------------------------------------------------
\begin{document}
%-----       resume       ---------------------------------------------------------
\makecvtitle
\cvitem{}{
}

Dear Sir/Madam,
\newline

%%%%%%%%% applying and blabla give me the job
I am writing to apply for the indefinite Experimental Physicist position at CERN.
Having been a member of the particle physics field for more than ten years I am committed to deepening our knowledge of fundamental physics, and have extensive experience in many aspects of experimental particle physics. I believe that a permanent position at CERN would be the ultimate way for me to make a lasting contribution to the field during the LHC and HL-LHC era, and an exciting opportunity to shape the future of the next generation of experiments.
\newline

%%%%%%%%% short intro of myself and set the scene to describe the hierarchy problem
I am currently working on the ATLAS experiment as CERN LD staff, and previously as CERN fellow. During my period in ATLAS I have performed measurements and searches within the Top, Higgs, Exotics and SUSY groups. I have also contributed strongly in the Tile calorimeter and in the Trigger groups. During this time I have been entrusted with management and coordination roles with an increasing level of responsibility and leadership. 
\newline

%%%%%%%%% Intro and hierarchy problem
After the successful observation of the Higgs boson in Run 1, the increased energy of Run 2 provided the LHC experiments with an unprecedented dataset to explore the energy frontier. No significant excess has been observed so far, and stringent limits have been set on simplified models.
The motivation for some form of new physics is still strong, but it is obvious that it is not manifested in the vanilla signatures that were the main focus of attention.
The Higgs boson mass has been measured at the electroweak scale, and the corresponding \emph{hierarchy problem} remains an unsolved question that I would like to address.
\newline

%%%%%%%%% Short statement about my past activity around the hierarchy problem
Throughout my career, I have worked on physics analyses targeting different aspects of the hierarchy problem. I was the lead analyser for the first searches for t\=tH, which led to the confirmation of the SM-like nature of the top yukawa coupling. I then transitioned to searches for beyond-the-standard-model top partners, that would cancel the top-loop corrections to the higgs mass. I developed the first search for vector-like tops decaying to higgs bosons, covering for the first time the full spectrum of possible decays. I have also led a series of searches for scalar tops (stops) in the 1-lepton final state, with increasingly comprehensive coverage of multiple SUSY scenarios.
\newline

%%%%%%%%% current and short-term future, challenging higgsinos
Currently, my main physics analysis interest lies in searches for natural supersymmetry, in particular in the most challenging final states. 
I have developed an analysis targeting higgsino production with RPV decay to quarks, exploiting the final state with at least one lepton and very high jet and $b$-jet multiplicity. I developed new methods to estimate the backgrounds in this extremely complex final state. This search provides the first limits on hadronic higgsino decays since LEP. 
In the short-term future, my goal is to develop a consistent program to cover the natural region of light higgsinos, focusing on poorly covered and challenging final states. Some examples would be final states with no missing energy, long decay chains leading to high multiplicity of soft objects, or displaced decays. 
%Dedicated efforts to identify additional sensitivity gaps 
\newline

%%%%%%%%% working through others
As convener of two different physics subgroups, I have driven multiple analysis to increase their scope and be more ambitious. XXXXXX
\newline


%%%%%%%% trigger
Alongside physics analysis, I have been heavily involved in the trigger group, where I have contributed to the operation, development and design of the trigger menu, which defines the data that is recorded by ATLAS, and therefore what physics can be done with this data. I was appointed as trigger menu coordinator, and I had the pleasure of defining the Run 3 ATLAS trigger menu. In doing so, I successfully argued for a strongly increased recording rate for Run 3, which was endorsed by the experiment. At the same time I performed a critical review of the menu composition and was able to reduce rate and trigger CPU consumption in multiple areas with no impact on the physics program.
%Delayed stream?
\newline

The phase-I upgrades of the TDAQ system will equip the trigger with more resources to improve its performance. It is critical that we fully explore the capabilities of the upgrades and do not settle on the straightforward gains obtained by an improved technology. 
I am currently coordinator of the Level1-calorimeter algorithm and performance forum, where I am leading efforts to develop and optimize reconstruction and identification algorithms, in order to achieve the best possible performance out of the upgraded subsystems.
\newline

%%%%%%%%% Mid-term future, HL-LHC and trigger
In the future, I plan to continue to work on new physics searches at ATLAS, as well as contributing to the trigger system. 
I plan to take a leading role in the construction and commissioning of the TDAQ phase-2 system.
With my expertise I see myself well suited to make substantial contributions to the developments needed for these projects. Not only to bring them to a timely completion, but also to ensure that they lead to the largest possible gains on the physics program. 
It is of the utmost importance that we, as a community, fully 
capitalize the physics potential of the high-luminosity LHC, which represents a huge international 
investment in fundamental physics research.
\newline

%%%%%%%%% Long-term future, next-gen collider
Looking further into the future, I would be excited to become involved in the physics opportunities of the next-generation particle collider, where several machines are under discussion. There is clearly a strong physics case for a Higgs factory, but also for a hadron collider at the energy frontier, though on a longer timeline. Personally, I find the full FCC program attractive for its rich physics potential. 
I would greatly enjoy the opportunity to become heavily involved in the design of a future detector.
\newline

%%%%%%%Conclusion
I am fully committed to deepening our knowledge of fundamental physics.
%I have extensive experience in many aspects of experimental particle physics. 
Throughout my career, I have pursued and implemented new and innovative ideas, and I have strived to balance my involvement across multiple areas of physics analysis, operations and detector. 
%This is evidenced through the broad range of activities detailed on my CV, and by the increasing level of leadership. 
CERN will clearly be at the heart of particle physics in the coming decades and 
therefore I would very much relish the opportunity to use my experience and 
abilities to further its physics program.
\newline


Sincerely,
\vfill
Javier Montejo Berlingen



\end{document}


%% end of file `template.tex'.
