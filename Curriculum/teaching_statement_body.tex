\section{Teaching statement}

%%%%%%%%% Teaching in broad sense, I like teaching
\cvitem{}{
Teaching in its broadest sense is an integral part of my daily commitments as a leading researcher in the ATLAS collaboration, and an incredibly satisfactory one. 
During my career I have had the privilege to attract and supervise some outstanding students, post-docs, and fellows. The engagement and creativity of young scientists in my analysis teams has clearly been one of the pillars leading to strong and innovative results.
Besides the rewarding outcome of seeing students mature into colleagues, I also enjoy the process and the teaching-learning dynamic of challenging and being challenged. 
One of my main goals is always to enable my students and foster their self-efficacy and problem-solving skills.
Thus, whenever possible I try to provide them with the support to answer their questions by themselves.
}

%%%%%%%%% No formal lecturing so far, but mentoring
\cvitem{}{
Due to my affiliation with CERN during the last seven years, I did not have the opportunity to give lectures in the most traditional sense.
However, due to my interest in contributing to the education and engagement of the next generation of young scientists, I have taken part in several mentoring and educational programs.
I have been discussion leader at the 2021 CERN-Fermilab Hadron Collider Physics Summer School, moderator at the International Masterclass 2016--2018, supervisor of CERN summer students, and assistant at the Beamline for schools program.
These experiences have been incredibly valuable, and have required me to adapt my approach in order to cater to the different needs of high-school, undergraduate, or PhD students.
}

%%%%%%%% My take on teaching
\cvitem{}{
My teaching philosophy originates from the years of observation and analysis of my own teachers, comparing their
different approaches, thinking about their teaching methods, and assessing
which methods enhanced my own learning and which ones hindered it.
These experiences as student, combined with the lessons I
am still learning from my own teaching experience, yields the two overriding principles that I strive
for in the classroom: \textbf{clarity} and the need for an \textbf{active involvement} of the teacher in the learning process. 
}

\cvitem{}{
\textbf{Clarity.} My own experience, both as a student and as a teacher, suggests that students who are
unclear about expectations often get frustrated and tend to resist learning. I strive to
be extremely clear in presenting material, in detailing expectations, and in
expressing educational goals.
My goal in a lecture, a presentation, or simply addressing a question, is that each student develops an internal understanding and intuition of the
concept being discussed, and couples that picture to the mathematical language by which
we study, explain, and predict high energy physics. 
To accomplish this, the explanations I provide must be very clearly presented, to allow students to follow this dual path of developing an internal understanding and mastering the mathematical formalism that is used to express those ideas.
This standard requires students to know not just the specific details of the
physics we have covered, but also how those details connect to the broader concepts
that have been covered previously. In other words, I expect them to see the forest and the trees. 
}

\cvitem{}{
Besides defining the teaching goal of each lecture well and how it fits into the bigger picture of the entire course, it is also beneficial to communicate it to the students: knowing what one is supposed to learn, helps to learn it.
Exercises should directly relate to the lecture and present an opportunity to apply the concepts and strategies learned.
}

%%%%%%%%% Two examples, lack of engagement, and shyness to ask,
\cvitem{}{
\textbf{Active involvement.} Teaching is a form of communication, and as such both sides need to give and seek feedback to understand if the learning process is being successful.
A teacher should strive to deliver great lectures, but also take an active involvement in understanding if the lecture is meeting the expected goal, and if the students are actually learning from it.
}

\cvitem{}{
I noticed one common and often occurring issue that hinders learning.
Students do not admit when they do not know or understand something, nor ask for help when they get stuck solving a problem.
They are afraid they could leave a negative impression on their fellow students and supervisors/teachers.
Therefore, I find it very important to foster an environment that acknowledges that learning and research entail not knowing while striving at reducing it.
While teaching I make certain the students know that it will be necessary to ask questions, and that I expect them to do so.
}

\cvitem{}{
A second, related, issue is the lack of reaction by teachers to such kind of problems. In my personal experience, a complete silence with no questions after a lecture was never a sign that the whole classroom had fully grasped the concepts and topics that were explained. But rather a lack of engagement, shyness, and after some weeks of lectures a too large gap in knowledge to the material being covered. I have always tried to follow up an explanation with questions to gauge if it has been fully understood, and if needed provided further clarification of the possible nuances. Very often I found that despite my best efforts for clarity, previous misconceptions or simply confusions with similar concepts have derailed the explanation.
Therefore, while teaching I always take responsibility for making sure that the lecture or explanation is having the desired effect, and I react accordingly if that is not the case.
}

\cvitem{}{
In summary, I am eager to become an inspiring lecturer who delivers research-oriented education effectively, and helps students mature into thinking critically and working independently. 
%My past experience and observations have lead me to recognise the importance of the teacher's active involvement in making sure the commnunication is successful, and adapting the message, tools and methods accordingly if needed.
My past experience and observations lead me to adhere to a principle of clarity which demands that my lectures, expectations of students, and educational
goals for them be as clear as possible in the interest of maximizing their educations. And I recognize the importance of the teacher's active involvement in making sure the communication is successful, and adapting the message, tools and  methods accordingly if needed.
My main teaching goal is to create an effective learning environment that will help student acquire both problem solving skills and a deep conceptual understanding of the subject. 
}
