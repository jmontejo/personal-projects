
\documentclass[11pt,a4paper,sans]{moderncv}        % possible options include font size ('10pt', '11pt' and '12pt'), paper size ('a4paper', 'letterpaper', 'a5paper', 'legalpaper', 'executivepaper' and 'landscape') and font family ('sans' and 'roman')

% moderncv themes
\moderncvstyle{classic}                            % style options are 'casual' (default), 'classic', 'oldstyle' and 'banking'
\moderncvcolor{blue}                               % color options 'blue' (default), 'orange', 'green', 'red', 'purple', 'grey' and 'black'
%\renewcommand{\familydefault}{\sfdefault}         % to set the default font; use '\sfdefault' for the default sans serif font, '\rmdefault' for the default roman one, or any tex font name
%\nopagenumbers{}                                  % uncomment to suppress automatic page numbering for CVs longer than one page

% adjust the page margins
\usepackage[scale=0.75]{geometry}
\setlength{\hintscolumnwidth}{2.6cm}                % if you want to change the width of the column with the dates
%\setlength{\makecvtitlenamewidth}{10cm}           % for the 'classic' style, if you want to force the width allocated to your name and avoid line breaks. be careful though, the length is normally calculated to avoid any overlap with your personal info; use this at your own typographical risks...

\usepackage{color}
\usepackage{bm}
\usepackage{amstext}
\newcommand{\coloredLink}[2]{\textcolor{blue}{\href{#1}{#2}}}

\newcommand\ttbb{\ensuremath{t\bar{t}b\bar{b}}}
\newcommand\ttbar{\ensuremath{t\bar{t}}}
\newcommand\tttt{\ensuremath{t\bar{t}t\bar{t}}}
\newcommand\ttH{\ensuremath{t\bar{t}H}}
\newcommand\bbbar{\ensuremath{b\bar{b}}}
\newcommand{\met}{\ensuremath{E_{{T}}^{{miss}}}}
\newcommand{\pt}{\ensuremath{p_{T}}}

\newif\ifAddReferences  %% References
\newif\ifAddStatement  %% Statement of research interest
\newif\ifAddInternalTalks  %% References
\AddReferencestrue
\AddStatementfalse
\AddInternalTalksfalse



% personal data
\name{Javier}{Montejo Berlingen} %many FIXME around the CV, find them
%%%%%%%%%%%%%%%%%%%%%%
% SINCE YOU ARE READING THIS, LET ME REMIND YOU ABOUT THE IDEA OF HAVING A VERTICAL LINE IN THE MIDDLE
% AND THE RESULTS AND POSITIONS ON BOTH SIDES TO GIVE CONTEXT OF THE TIMELINE
%                                                                  CHECK 'CV nuevo layout.key' <----------
%%%%%%%%%%%%%%%%%%%%%%
%\title{title}                               % optional, remove / comment the line if not wanted
\address{CERN 40/5-C11}{1217 Meyrin}{Switzerland}% optional, remove / comment the line if not wanted; the "postcode city" and "country" arguments can be omitted or provided empty
%\phone[mobile]{+39~339~311~4119}                   % optional, remove / comment the line if not wanted; the optional "type" of the phone can be "mobile" (default), "fixed" or "fax"
%\phone[fixed]{+41~78~631~45~62}
\phone[fixed]{+41~786314562}
%\phone[fax]{+3~(456)~789~012}
\email{jmontejo@cern.ch}                               % optional, remove / comment the line if not wanted
%\homepage{www.johndoe.com}                         % optional, remove / comment the line if not wanted
%\social[linkedin]{francesco-rubbo}                        % optional, remove / comment the line if not wanted
%\social[twitter]{jdoe}                             % optional, remove / comment the line if not wanted
%\social[github]{francescorubbo}                              % optional, remove / comment the line if not wanted
%\extrainfo{additional information}                 % optional, remove / comment the line if not wanted
%\photo[64pt][0.4pt]{picture}                       % optional, remove / comment the line if not wanted; '64pt' is the height the picture must be resized to, 0.4pt is the thickness of the frame around it (put it to 0pt for no frame) and 'picture' is the name of the picture file
%\quote{some quote}                                 % optional, remove / comment the line if not wanted

% to show numerical labels in the bibliography (default is to show no labels); only useful if you make citations in your resume
%\makeatletter
%\renewcommand*{\bibliographyitemlabel}{\@biblabel{\arabic{enumiv}}}
%\makeatother
%\renewcommand*{\bibliographyitemlabel}{[\arabic{enumiv}]}% CONSIDER REPLACING THE ABOVE BY THIS

\usepackage{multibib}
\newcites{article,confnote,proceedings}{{Articles},{Conference Notes},{Proceedings}}
%----------------------------------------------------------------------------------
%            content
%----------------------------------------------------------------------------------
\begin{document}
%-----       resume       ---------------------------------------------------------
\makecvtitle

\section{Personal Information}
\cvline{Date of birth}{March 26, 1986}
\cvline{Place of birth}{Salamanca, Spain}
\cvline{Sex}{Male}
\cvline{Nationality}{Spanish, German}

\setlength{\hintscolumnwidth}{6.2cm}%{7.3cm} 
\setlength{\maincolumnwidth}{\textwidth-\leftskip-\rightskip-\separatorcolumnwidth-\hintscolumnwidth}
\section{Education and Research Positions}
%line breaks in first field require \\, and \newline in the rest of fields
\cventry{\textbf{CERN LD Staff (2017 - today)}}{}{}{}{}{}
\cventry{SUSY RPV and long-lived subgroup convener\\Trigger menu coordinator\\Trigger coordination scientific secretary}{}{}{}{
JHEP 05 (2019) 142\newline{\color{black!70}\small{Constraints on mediator-based dark matter and scalar dark energy models using $\sqrt{s}=13$ TeV $pp$ collision data collected by the ATLAS detector.}}\normalsize{~}
\newline
ATLAS-CONF-2018-003\newline{\color{black!70}\small{Reinterpretation of searches for supersymmetry in models with variable R-parity-violating coupling strength and long-lived R-hadrons.}}\normalsize{~}
\newline
JHEP 06 (2018) 108\newline{\color{black!70}\small{Search for top-squark pair production in final states with one lepton, jets, and missing transverse momentum using 36 fb$^{-1}$ of $\sqrt{s}=13$ TeV $pp$ collision data with the ATLAS detector.}}\normalsize{~}
}{}
\cventry{\textbf{CERN fellowship (2015 - 2017)}}{}{}{}{}{}
\cventry{SUSY third generation subgroup convener}{}{}{}{
JHEP 09 (2017) 088\newline{\color{black!70}\small{Search for new phenomena in a lepton plus high jet multiplicity final state with the ATLAS experiment using using $\sqrt{s}=13$ TeV $pp$ collision data}}\normalsize{~}\newline 
ATLAS-CONF-2016-050\newline{\color{black!70}\small{Search for top squarks in final states with one isolated lepton, jets, and missing transverse momentum in$\sqrt{s}=13$ TeV $pp$ collisions with the ATLAS detector (14 fb-1)}}\normalsize{~}\newline
Phys. Rev. D 94, 052009\newline{\color{black!70}\small{Search for top squarks in final states with one isolated lepton, jets, and missing transverse momentum in$\sqrt{s}=13$ TeV $pp$ collisions with the ATLAS detector (3.2 fb-1)}}\normalsize{~}
}{}


\cventry{Aug 2015}{CERN Fellowship}{}{}{\newline{}.}{}
\cventry{Jun 2015}{Ph.D.,}{Universitat Aut\'onoma de Barcelona,}{Spain.}{\newline{}Thesis: Searches for new physics in final states with \ttbar \ and additional heavy flavour jets.\newline
ATLAS thesis award, Springer thesis award.}{}
\cventry{Feb 2012}{M.Sc. in high energy physics,}{Universitat Aut\'onoma de Barcelona,}{Spain.}{\newline{}Master thesis: Timing performance of the Tile calorimeter with muons from collision events.}{}
\cventry{Jan 2011}{B.Sc. in Computer Science,}{Universidad de Salamanca,}{Spain.}{\newline{}
Extraordinary Graduation Award. \newline{}B.Sc. thesis: jSchroedinger, multiplatform application for the visualisation and numerical resolution of quantum potentials.}{}
\cventry{Jan 2010}{B.Sc. in Physics,}{Universidad de Salamanca,}{Spain.}{\newline{}Extraordinary Graduation Award.\newline{}National Award for Excellence in Academic Performance.}{}{}

\cventry{Jan 2010}{B.Sc. in Physics,}{Universidad de Salamanca,}{Spain.}{\newline{}Extraordinary Graduation Award.\newline{}National Award for Excellence in Academic Performance.}{}{}


\setlength{\hintscolumnwidth}{2.6cm} 
\setlength{\maincolumnwidth}{\textwidth-\leftskip-\rightskip-\separatorcolumnwidth-\hintscolumnwidth}
\clearpage
\section{Research Experience}

\label{subsec:validation}
\cventry{\textbf{SUSY in Run~II}}{}{}{}{
The large increase in energy and luminosity at the start of Run~II made it the perfect moment to embark in BSM searches, where my main interest lies in SUSY models. I am currently working in the search for direct stop production, where we published two early results [\coloredLink{http://journals.aps.org/prd/abstract/10.1103/PhysRevD.94.052009}{Phys. Rev. D 94 (2016) 052009}] [\coloredLink{https://atlas.web.cern.ch/Atlas/GROUPS/PHYSICS/CONFNOTES/ATLAS-CONF-2016-050}{ATLAS-CONF-2016-050}]. A paper with the full 2015-2016 dataset is in preparation, where the increased dataset allows to study additional models, like the decay of stops to higgsinos. The small mass splitting among the higgsinos produces very low \pt\ leptons, leading to a very interesting and challenging final state. [\coloredLink{https://link.springer.com/article/10.1007/JHEP06(2018)108}{JHEP 06 (2018) 108}]
}{}{}
\cventry{}{}{}{}{
Although supersymmetry is a very compelling extension of the SM, no hints for vanilla SUSY has been observed. This motivates the extension of the searches to less traditional final states. I have developed a completely new search for RPV SUSY in final states with no missing transverse energy (\met), at least one lepton and high number of jets. The absence of \met\ in the final state is a feature that could cause supersymmetric particles to remain unobserved due to the large requirements on \met\ placed by most searches. The background modeling at very high jet multiplicities is extremelly challenging, and new data-driven methods were developed to estimate the background. The resulting analysis has been published [\coloredLink{https://link.springer.com/article/10.1007\%2FJHEP09\%282017\%29088}{JHEP 09 (2017) 088}] and also reinterpreted in other models due to its versatility and sensitivity to four-top production [\coloredLink{https://link.springer.com/article/10.1007/JHEP05(2019)142}{JHEP 05 (2019) 142}].
}{}{}
\cventry{}{}{}{}{
{\color{red}DISCUSS RPC-RPV, add taskforce chair to positions?} %FIXME
[\coloredLink{https://atlas.web.cern.ch/Atlas/GROUPS/PHYSICS/CONFNOTES/ATLAS-CONF-2018-003/}{ATLAS-CONF-2018-003}]
{\color{red}DISCUSS DM summary paper and 4top reinterpretation} %FIXME
}{}{}

\cventry{\textbf{ttH}}{}{}{}{The main topic of my Ph.D. work has been the search for the associated production of a Higgs boson and a top quark pair (\ttH), with the decay of the Higgs boson to \bbbar\ and semileptonic \ttbar\ decay. This process depends on the Yukawa coupling between the top quark and the Higgs boson at tree level, therefore it is sensitive to variations of the coupling, and allows also to disentangle the effect of new physics in loop-induced processes like gluon fusion to Higgs and H $\rightarrow \gamma\gamma$.
However, it is also an extremely challenging measurement due to the large backgrounds from \ttbar+jets, which suffers from large systematic and theoretical uncertainties, and the large combinatorial background from the high b-jet multiplicity.}{}{}
\cventry{\textbf{}}{}{}{}{
The results were summarized in two conference notes at 7 and 8 TeV, and a paper published in EPJC. \newline
[\coloredLink{https://atlas.web.cern.ch/Atlas/GROUPS/PHYSICS/CONFNOTES/ATLAS-CONF-2012-135/}{ATLAS-CONF-2012-135}]
[\coloredLink{https://atlas.web.cern.ch/Atlas/GROUPS/PHYSICS/CONFNOTES/ATLAS-CONF-2014-011/}{ATLAS-CONF-2014-011}]
[\coloredLink{http://dx.doi.org/10.1140/epjc/s10052-015-3543-1}{EPJC 75 (2015) 349}]
}{}{}

\cventry{\textbf{Vector-like quarks and $\bm{\tttt}$}}{}{}{}{
Vector-like top partners are a common feature of many SM extensions such as Little Higgs or extra-dimensional models. One of the allowed decays of these vector-like partners is $T \rightarrow H t$, which gives rise to a signal with high number of jets and b-tags. The same signature is also characteristic of \tttt\ final states, a final state which can receive large contributions from BSM models. \newline
Building on my experience with \ttH\ I performed the analysis for VLQ and \tttt. [\coloredLink{http://dx.doi.org/10.1007/JHEP08(2015)105}{JHEP 08 (2015) 105}]
 }{}{}
\cventry{\textbf{SUSY in Run~I}}{}{}{}{
Simplified supersymmetric models have been thoroughly studied and searched for, but some regions of the parameter space have not been excluded yet. One of the scenarios where the traditional SUSY searches have little sensitivity occurs when the lightest supersymmetric partner of the top quark (stop) is only slightly heavier than the sum of the masses of the top quark and the lightest supersymmetric particle (LSP).}{}{}
\cventry{\textbf{}}{}{}{}{
Targeting this scenario, I worked on a search for direct pair production of the heavy stop ($\tilde{t}_2$) decaying via $\tilde{t}_2 \rightarrow H \tilde{t}_1$ and a subsequent $\tilde{t}_1 \rightarrow t \tilde{\chi}_1^0$. 
The analysis was published in EPJC as part of the third generation summary paper.
[\coloredLink{http://dx.doi.org/10.1140/epjc/s10052-015-3726-9}{EPJC 75 (2015) 510}]
}{}{}
\cventry{\textbf{Trigger}}{}{}{}{The increase in peak luminosity and pileup has been a great challenge to the trigger system. As trigger menu expert and on-call I have contributed to the development and operation of a robust trigger menu. The development of complex algorithms and the pileup increase has pushed the HLT CPU farm to its limits. In order to guarantee a sustainable operation of the trigger during 2017 I have studied the pileup dependency and CPU consumption of the trigger menu.
{\color{red}Trigger Menu PUB note} %FIXME
}{}{}
\cventry{\textbf{Tile calorimeter}}{}{}{}{During my years in ATLAS I have also devoted a fraction of my time to working on the Tile calorimeter. My work was focused on the understanding and calibration of the timing performance of the calorimeter using collision data. Multiple detector and geometrical effects that were previously not accounted for were identified, and a set of selection cuts and corrections were introduced that improved the resolution of the time measurement up to 20\%. [\coloredLink{https://link.springer.com/article/10.1140/epjc/s10052-018-6374-z}{ATL-TILECAL-INT-2012-005}] \newline
I also gained some hardware experience by contributing to the maintenance and repairs of Tilecal modules during the LS1 phase.
After my work on the calorimeter performance I contributed as Data Quality Validator and Data Quality Leader for the Tile calorimeter.
}{}{}

\ifAddInternalTalks
\section{Conferences, schools and workshops}
\else
\section{Conference talks}
\fi
\cventry{May 2019}{27th International Conference on Supersymmetry and Unification of Fundamental Interactions (SUSY 2019),}{Corpus Christi,}{Texas}{\newline{}
    Talk: Searches for supersymmetry in R-parity violating signatures at the LHC.
    [\coloredLink{https://cds.cern.ch/record/2676781}{ATL-PHYS-SLIDE-2019-242}]
}{}{}
\ifAddInternalTalks
\cventry{May 2019}{Trigger workshop,}{Elba,}{Italy.}{\newline{}
    Talk: Run-3 baseline menu.}{}{} %https://indico.cern.ch/event/772409/
\cventry{Sep 2018}{TDAQ week,}{Krakow,}{Poland.}{\newline{}
    Talk: Menu considerations for Run 3.}{}{} %https://indico.cern.ch/event/730816
    \fi
\cventry{Jul 2018}{23rd International Conference on Computing in High Energy and Nuclear Physics (CHEP 2018),}{Sofia,}{Bulgaria.}{\newline{}
    Talk: The ATLAS Trigger Menu design for higher luminosities in Run 2. [\coloredLink{https://cds.cern.ch/record/2631630}{ATL-DAQ-SLIDE-2018-500}][\coloredLink{https://cds.cern.ch/record/2645347}{Proceedings}]
}{}{} 
\cventry{May 2018}{(Re)interpreting the results of new physics searches at the LHC (ReINPS2018),}{CERN,}{Geneva}{\newline{}
    Talk: Sensitivity of prompt searches to long-lives particles.
    [\coloredLink{https://cds.cern.ch/record/2319790}{ATL-PHYS-SLIDE-2018-282}]
}{}{}
\ifAddInternalTalks
\cventry{Apr 2018}{Four-tops workshop,}{CERN,}{Geneva}{\newline{}
    Talk: Data-driven methods in lepton+jets searches to evaluate the ttbar+jets and W+jets backgrounds.}{}{}
    \fi
\cventry{May 2017}{Large Hadron Collider Physics (LHCP),}{Shangai,}{China.}{\newline{}
    Talk: Unconventional signatures and RPV supersymmetry. Plenary. [\coloredLink{https://cds.cern.ch/record/2266308}{ATL-PHYS-SLIDE-2017-310}]
 \newline{}
    Poster: The ATLAS Run-2 Trigger Menu for higher luminosities: Design, Performance and Operational Aspects. [\coloredLink{https://cds.cern.ch/record/2265272}{ATL-DAQ-SLIDE-2017-255}] }{}{}
\ifAddInternalTalks
\cventry{May 2017}{ATLAS Exotics \& SUSY workshop,}{Bucharest,}{Romania.}{\newline{}
    Talk: RPV searches in ATLAS.}{}{}
\cventry{Feb 2017}{ATLAS trigger workshop,}{University of Geneva,}{Switzerland.}{\newline{}
    Talk: Trigger rates and processing time. Pileup dependency.}{}{}
\cventry{Oct 2016}{ttH workshop,}{CERN,}{Geneva.}{\newline{}
    Talk: ttH to invisible.}{}{}
\cventry{Sep 2016}{TDAQ week,}{Barcelona,}{Spain.}{\newline{}
    Talk: Trigger menu rates and CPU projections in 2017.}{}{}
\cventry{Apr 2016}{Supersymmetry workshop,}{Sussex,}{UK.}{\newline{}
    Talk: RPV analyses in ATLAS.}{}{}
    \fi
\cventry{Mar 2015}{XXIX Rencontres de Physique de la Vall\'{e}e d'Aoste,}{La Thuile,}{Italy.}{\newline{}
    Talk: Search for the Higgs boson in the ttH production channel using the ATLAS detector. 
    %\newline
     [\coloredLink{https://cds.cern.ch/record/2002318}{ATL-PHYS-SLIDE-2015-073}]
    }{}{}
\ifAddInternalTalks
\cventry{Nov 2014}{ttH workshop,}{CERN,}{Geneva.}{\newline{}
    Talk: Sensitivity projections for ttH.}{}{}
\cventry{Jun 2014}{European School of High-Energy Physics,}{Garderen,}{Holland.}{}{}
\cventry{Sep 2013}{ATLAS Higgs to \bbbar\ workshop,}{Marseille,}{France.}{\newline{}
    Talk: ttbar modeling for VH and ttH.}{}{}
    \fi
\cventry{Aug 2013}{21st International Conference on Supersymmetry and Unification of Fundamental Interactions (SUSY 2013),}{Trieste,}{Italy.}{\newline{}
    Talk: Top quark production in the ATLAS detector of the LHC. \newline
     [\coloredLink{https://cds.cern.ch/record/1596998}{ATL-PHYS-SLIDE-2013-509}]
}{}{}
\ifAddInternalTalks
\cventry{Jun 2013}{ttH workshop,}{CERN,}{Geneva.}{\newline{}
    Talk: tt+jets/bb modeling and systematics in ttH(bb) analyses.}{}{}
    \fi
\cventry{May 2013}{Large Hadron Collider Physics (LHCP 2013),}{Barcelona,}{Spain.}{\newline{}
	Poster: Search for the Standard Model Higgs boson produced in association with top quarks and decaying to $b\bar{b}$ at $\sqrt{s} = 7~TeV$. \newline
	[\coloredLink{https://cds.cern.ch/record/1551958}{ATL-PHYS-SLIDE-2013-298}]% \newline{}
   [\coloredLink{http://www.epj-conferences.org/articles/epjconf/abs/2013/21/epjconf_lhcp2013_20028/epjconf_lhcp2013_20028.html}{Proceedings}]}{}{}
\ifAddInternalTalks
\cventry{Dec 2012}{ATLAS Higgs to \bbbar\ workshop,}{Saint Genis,}{France.}{\newline{}Talk: Statistical procedures and limit setting in Higgs to \bbbar.}{}{}
\cventry{Jul 2012}{Hadron Collider Physics School (HASCO 2012),}{G\"ottingen,}{Germany.}{}{}
\fi
\cventry{Mar 2012}{LHC commissioning,}{CERN,}{Geneva.}{\newline{}Poster: ATLAS detector performance in 2011: calorimeters. \newline
[\coloredLink{https://cds.cern.ch/record/1445591}{ATL-PHYS-SLIDE-2012-154}]}{}{}
\ifAddInternalTalks
\cventry{Sep 2011}{4th International Workshop on Top Quark Physics (TOP 2011),}{Sant Feliu de Guixols,}{Spain.}{}{}
\fi
% FIXME this is more for a general job
%\section{Languages and Computer skills}
%\cvline{}{Spanish (Native), German (Native), English (Fluent, C2), French (Basic, B1)}
%\cvline{Programming}{C/C++, Python, Java, ROOT}

\clearpage

\ifAddStatement
\section{Statement of research interest}
\cventry{\textbf{}}{}{}{}{
After the observation of the Higgs boson, one of the main priorities for Run II of the LHC is the exploration of the energy frontier. No significant excess has been observed so far with the current Run II dataset, and stringent limits have been set on simplified models. 
The motivation for some form of new physics is still strong but it is obvious that it is not manifested in the vanilla signatures that were the main focus of attention. 
At this stage two different approaches can and should be pursued. One on side, searches in final states with low cross section or challenging signatures will become more relevant with the growing dataset. On the other side, precision measurements can provide a ground where deviations from the SM prediction can shed further light on the presence or absence of new physics.
}{}{}
% Higgs to bb
\cventry{\textbf{H $\bm{\rightarrow}$ bb}}{}{}{}{
%\cventry{\textbf{}}{}{}{}{
The discovery of the Higgs boson was a milestone in the LHC program, and although so far all the measurements are compatible with the SM prediction, most of its properties have not been measured yet. 
In particular the decay of the Higgs boson to b-quarks has not been observed, although it is the decay with highest branching ratio. The associated production of the Higgs with a vector boson is close to achieving the first evidence for such decay, and will be a key analysis to establish and measure the decay to b-quarks. Beside the intrinsic relevance of the VH channel, it is also a sensitive channel in which to look for BSM signals, for example in electroweak supersymmetry decays. This is an area where I would like to contribute, and where dedicated and detailed work can make a large impact towards a first observation.
}{}{}
% V+bb
\cventry{\textbf{V+bb}}{}{}{}{
%\cventry{\textbf{}}{}{}{}{
The large increase in energy and luminosity has been a great benefit to most analyses in ATLAS. However, the relative slow increase that is foreseen in the next years will demand an improvement in the understanding of background modeling and systematics in order to surpass significantly the current level of sensitivity. In particular the precise measurement of V+bb will be a key ingredient towards the observation of VH. Measurements of vector boson in association with b-quarks have been performed in Run 1. However, they don't profit from the increased beam energy available in Run 2 and with little focus on the collinear regime, where the two b-quarks are close and can merge into a single jet. This regime is both challenging and very interesting since MC generators are usually not able to model it correctly.
}{}{}
% gluon to bb
\cventry{\textbf{gluon $\rightarrow$ bb}}{}{}{}{
%\cventry{\textbf{}}{}{}{}{
The production of b-quarks from gluon splitting is a background faced by almost every analysis requiring b-jets in the final state, and the ability to identify such splittings would immediately improve many searches and measurements. Several implementations for such a tagger have been designed, however focusing mostly on the high-p$_{\text{T}}$ regime or specifically on Higgs decays. Given the multiple applications, I would like to work towards the development of a dedicated gluon to bb tagger, focusing on the low and intermediate p$_{\text{T}}$  regime. Any improvement in gluon identification will directly improve the aforementioned analyses, and can also be of wider interest to the whole collaboration. 
}{}{}
\fi
\clearpage

\ifAddReferences
\section{References}

\cventry{}{ \textbf{Andreas Hoecker} }{}{\newline
European Organization for Nuclear Research (CERN) \newline
+41 22 76 74787\newline
andreas.hoecker@cern.ch\newline
}{}{}

\cventry{}{ \textbf{Marumi Kado} }{}{\newline
Laboratoire de l'Accelerateur Lineaire (LAL)\newline
+41 22 76 71143\newline
kado@lal.in2p3.fr\newline
}{}{}

\cventry{}{ \textbf{Aurelio Juste Rozas} }{}{\newline
Institut de Fisica d'Altes Energies (IFAE)\newline
+34 93 581 4249\newline
juste@ifae.es\newline
}{}{}

\cventry{}{ \textbf{Brian Petersen} }{}{\newline
European Organization for Nuclear Research (CERN) \newline
+41 22 76 71199\newline
brian.petersen@cern.ch\newline
}{}{}

\cventry{}{ \textbf{Joerg Stelzer} }{}{\newline
University of Pittsburgh\newline
+41 22 76 78979\newline
stelzer@cern.ch\newline
}{}{}

\fi


\end{document}


%% end of file `template.tex'.
