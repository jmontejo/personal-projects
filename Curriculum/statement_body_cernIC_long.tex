\section{Statement of research interest}

%%%%%%%%% Intro and hierarchy probelm
\cvitem{}{
After the successful observation of the Higgs boson in Run 1, the increased energy of Run 2 has provided the LHC experiments with an unprecedented dataset to explore the energy frontier. No significant excess has been observed so far, and stringent limits have been set on simplified models.
The motivation for some form of new physics is still strong, but it is obvious that it is not manifested in the vanilla signatures that were the main focus of attention.
The Higgs boson mass has been measured at the electroweak scale, and the corresponding \emph{hierarchy problem} remains an unsolved question that I would like to address.
}

%%%%%%%%% Natural supersymmetry and RPV
\cvitem{}{
Natural supersymmetry remains a prime candidate to solve the hierarchy problem, and in particular R-parity violating (RPV) supersymmetry models can feature naturally light mass spectra with only weak exclusion limits. One of the key ingredients in natural supersymmetry is the presence of light higgsinos. Searches for higgsino production in final states with missing energy are well established at the LHC, and will continue progressing. However, higgsino searches without missing energy, as predicted by RPV, are largely uncovered. My goal is to develop a program to cover the natural region of light higgsinos, including the most challenging decays.}
%Ewk production slepton-mediated > 1 TeV

%%%%%%%%% multi-lepton multi-bjet and LQD
\cvitem{}{
I have developed an analysis targeting higgsino production with RPV decay via a baryon-number-violating coupling $UDD$, exploiting the final state with at least one lepton and many $b$-jets ($\geq 1 \ell, \geq 3$ b-jets).
This search provides the first limits on hadronic higgsino decays since LEP. I would like to extend the analysis into the multi-leptons multi-$b$-jets final state ($\geq 3 \ell, \geq 3$ b-jets), in order to target also effectively decays via the lepton-number-violating coupling $LQD$, which is so far uncovered. 
Supersymmetric models with this coupling have received renewed interest recently given their potential to explain the $R_K$, $R_K^*$, and muon $g-2$ anomalies.
}

%%%%%%%%% ttW+bb measurement
\cvitem{}{
This final state is certainly not new, and searches have already been performed, focusing however on high-mass signals. 
For low-mass signals, the lack of striking distinguishing features requires an excellent control of the background modelling in order to fully capitalise on the possibilities of this channel.
The main background in this final state originates from \ttW\ with additional heavy flavour jets. %, and to a lesser extent \ttbar\ with additional fake or non-prompt leptons. 
Despite the availability of NLO predictions for the \ttW\ process, the modelling uncertainties on this background remain large, and several analyses observe data/MC disagreements above 50\%. Keeping in mind the long-term possibilities of this final state, I consider that a differential measurement of the \ttW\ process, in particular with additional heavy-flavour jets, is a required first step to establish the program.
}

%%%%%%%%% BSM in 3l3b
\cvitem{}{
With backgrounds under control, a search for higgsinos decaying via $LQD$ can be performed. 
%Besides the usual search for excesses in certain regions of phase space, a search for asymmetries in lepton flavours can also target the most motivated couplings, and provide 
It is worth highlighting that searches in multi-lepton and multi-$b$-jets final states are not exclusive to RPV supersymmetry. Other RPC models such as stealth supersymmetry, NMSSM, or longer decay chains within the electroweak superpartners also predict similar final states. %, as well as non-supersymmetric models such as 2HDMs or vector-like quarks. 
A common aspect of these models is the possibility for higgsinos to decay via multiple vector bosons and higgs bosons or additional scalars, depleting the final state from \met. As part of the program, a search for b\=b resonances in this final state would tackle these kind of models. 
%Therefore, light higgsinos with such challenging decays have not been tested yet, and could also be accessed with the multi-lepton multi-$b$-jet final state.
Therefore, the multi-lepton multi-$b$-jet final state can be used to target a large variety of models featuring light higgsinos with challenging decays, that have not been tested so far.
}

%%%%%%%%% long-lived 
\cvitem{}{
In order to achieve an exhaustive coverage of the higgsino landscape, a possibility that should not be neglected is that the higgsinos could be long-lived. 
%The numerical values of the RPV couplings are often considered a free parameter of the theory. 
A variety of models that include RPV terms also predict very small values for the couplings, which lead to displaced decays. Searches for displaced decays, such as displaced vertices, have been conducted at the LHC, however focusing mostly on strong production. 
The selections and reconstruction techniques that target gluinos and R-hadrons above 2 TeV are extremely inefficient in the search for higgsinos at the electroweak scale. %they might be familiar with H 4b -like DVs
%Searches for displaced decay products from low-mass scalars also exist, but rely on the associated production with a vector boson, which is not possible in the case of higgsinos.
In addition, the background levels increase dramatically when attempting to adapt current searches towards vertices with lower masses and lower number of tracks, as required to have acceptance to lower mass scales.
Therefore, I would like to pursue a program for electroweak-scale displaced decays, leading to displaced vertices. 
%And in particular, focusing on processes such as higgsino production, that do not benefit from the possibility of associated production with a vector boson. 
}

%%%%%%%% DV with soft muons
\cvitem{}{
From an experimental point of view, one of the main handles to identify the signals previously discussed was the presence of a high number of $b$-jets. This feature is however seemingly lost when considering displaced decays. One possibility to recover it however, is to exploit B-hadron decays to muons. By requiring a displaced vertex with one or more muons attached, the backgrounds can be massively reduced, enabling a search for displaced vertices with lower masses and lower number of tracks. Other interesting signals that can be accessed lowering the track multiplicity requirements are for example decays to two displaced taus. The low production cross section and additional penalty from the muonic branching ratio of B-hadrons requires a large recorded dataset, but the achieved purity can be excellent, making this a suitable target for the upcoming LHC runs. 
%
}

%%%%%%%% trigger
\cvitem{}{
One of the main challenges of such program is the absence of an effective trigger. The final state contains no significant \met, the low mass scale that is targeted is significantly below the hadronic trigger thresholds, and possible leptons in the decay are displaced, rendering the usual lepton triggers inefficient. Many searches rely on the associated production with a vector boson to trigger. However, this option is not present in the case of higgsino production, as well as other models, and other trigger strategies need to be developed.
Two different approaches can and should be pursued. One on side, triggering directly on the displaced decay products can provide a high efficiency at the cost of some model-dependence. In particular I would like to develop a trigger for displaced vertices with additional displaced leptons in the vertex. The leptons can originate either from the BSM particle, top quarks, or b-quarks in the decay. The lepton provides handles to reduce the rate, and defines a region in the detector where the displaced vertex reconstruction can be run. Limiting therefore the rate and area where the CPU-intensive reconstruction of large-radius tracks has to run.
The second approach to triggering would be to rely on the products of some associated production. The associated production with a vector boson is not possible for a higgsino signal, however the production via vector-boson-fusion can be exploited for this purpose. Triggering on a high-mass dijet system with large separation in $\eta$ would allow to reconstruct and trigger also on displaced vertices without leptons, opening up possibilities to target also low-mass signals without leptons or heavy-flavour quarks.
%HGTD?
}

%\cvitem{}{
%The ATLAS measurement program has been extremely successful, and processes with cross-sections as low as $\mathcal{O}(1)$ fb have been measured. However it is important to keep in mind that so far less than 5\% of the target of 3000 $\text{fb}^{-1}$ has been recorded. The large increase in luminosity will benefit especially processes with low cross-sections, and final states with high purity but low branching ratios, such as multi-lepton final states. In addition, it will also transform the treatment of objects where the choice of working point means a trade-off between purity and statistics. Examples of such are identification and isolation of leptons, or tagging of jets originating from $b$-quarks ($b$-jets). 
%For the aforementioned reasons, I consider that a program of measurements in final states with multiple leptons and $b$-jets will become extremely valuable during the next years, and up to the high-luminosity phase of LHC (HL-LHC). 
%}

%%%%%%%Conclusion
\cvitem{}{
As discussed, I consider that a program of searches for higgsinos with challenging decays is needed to cover as exhaustively as possible one of the key predictions of natural supersymmetry. Searches in final states with multiple leptons and $b$-jets have a large potential to explore such challenging decays, although precise measurements of the main background will be needed to fully capitalise on the possibilities of this final state. Extending such searches into the regime of long-lived decays is difficult but certainly possible. Dedicated triggers will need to be developed to effectively record such events. In both prompt and displaced searches, the lepton and $b$-jet multiplicities and identification criteria offer a continuous trade-off between purity and available statistics. Therefore proving the flexibility to fully exploit the fast increase in recorded dataset during the next years, and up to the high-luminosity phase of LHC.}
	
%\cvitem{}{
%\color{red} I'm a bit concerned about the balance between what I want to do and why I want to do it. I feel like I might be overemphasizing the discussion about models that can be targeted an so on. Also the interplay with existing searches and why those can't be used. I'm obviously not a hardware person, but I feel like I could highlight a bit more the impact of the planned upgrades.}
