\section{Statement of research interest}
\cventry{\textbf{}}{}{}{}{
After the observation of a particle compatible with the Higgs boson, and the lack of direct evidence of physics beyond the Standard Model, the main priority for Run II of the LHC will be the exploration of the energy frontier. Although the Standard Model is now consistent up to the Planck scale, the presence of unexplained observations such as neutrino masses, dark matter or the matter-antimatter asymmetry require the presence of new physics at an intermediate scale.
}{}{}
\cventry{\textbf{}}{}{}{}{
In many new physics scenarios, hypothetical new particles will couple preferentially to the top quark, owing to its large mass. Therefore, one of my interests is to perform searches for new physics in top-like final states. In particular ttbb and four tops final states are very promising signatures in the search for new physics. The search for vector-like quarks, contact interactions and gluino pair production are some examples of the searches that can be carried out in these final states. The experience I obtained during Run I in high jet and high b-tag final states, are crucial for such searches.
}{}{}
\cventry{\textbf{}}{}{}{}{
A related line of research is searches in boosted topologies, which will play a prominent role during Run II. The increase in energy will enhance significantly the fraction of events where the decay products are boosted and collimated, therefore possible to be reconstructed in a single ''fat'' jet. The development of sophisticated tools to identify new massive particles by studying the substructure of reconstructed ''fat'' jets will increase the sensitivity of these searches. New resonances decaying to top pairs are one of the multiple new physics scenarios where the usage of substructure techniques can enhance the sensitivity. Given the relevance and the many possibilities of boosted techniques, in conjunction with my expertise in top-like final states, this is a topic in which I am really interested.
}{}{}
\cventry{\textbf{}}{}{}{}{
The increased energy and instantaneous luminosity will not only bring an incredible increase in sensitivity for new physics, but also new experimental challenges. One of the topics that I would like to address is b-tagging in dense environment, and for high energy jets. Although the performance of the b-tagging algorithms has steadily increased during Run I, a higher inefficiency to b-tag jets with high transverse momentum has always been present. 
%Recent developments like the MVb algorithm tried to address such problems. 
Given the importance of this topic I would like to contribute to the improvement of the flavour-tagging performance. In the context of searches for new physics in \ttbar\ final states, this contribution becomes even more relevant, since the presence of b-jets with high transverse momentum is a distinctive signature.
}{}{}
\cventry{\textbf{}}{}{}{}{
In summary, I believe my academic training at the Universidad de Salamanca, and at
IFAE, my contribution to different analysis and my work in detector performance, have prepared me to be an effective postdoctoral researcher. 
%Your institution, hosting one of the largest and most prestigious ATLAS groups, would be the ideal place to continue my research.
I would be happy to answer any further questions you may have, or provide any supplementary material you may need. I thank you very much for your consideration of my candidacy and sincerely look forward to hearing from you.}{}{}

