\chapter{Conclusions}
\label{chapter:Conclusions}

This dissertation presents searches in \ttbar\ final states with additional heavy-flavor jets using $\unit[20.3]{fb^{-1}}$ of \pp\ collision data at 
$\sqrt{s} = \unit[8]{\TeV}$, recorded with the ATLAS experiment at the LHC. 
Exploiting this final state, three analyses are presented that  probe the stability of the Higgs boson mass from different perspectives.

The main challenge for the analyses presented lies in the precise modeling of the background, in particular \ttbb\ production. Since no differential measurements have been performed yet on the \ttbar\ production with additional heavy-flavor jets, the analyses have to rely on MC simulation for the background. Recent developments in MC simulation have improved the description of the background, and a great effort is invested in porting the state-of-the-art predictions into the analyses. The systematic uncertainties on the modeling of the \ttHF\ background constitute the main source of sensitivity degradation.
In order to reduce the impact of the systematic uncertainties, both theoretical and experimental, the analyses use a profile likelihood fit exploiting high-statistics control regions to constrain in-situ the leading uncertainties and improve the background modeling.

The first of the analyses aims to study the \ttH\ process and to measure its production rate, from which the top Yukawa coupling can be extracted. Neural networks are used to discriminate the \ttH\ signal from the background, and variables computed using matrix element method are included in the training.
No evidence for the \ttH\ process is found, and a \unit[95]{\%} CL upper limit is set, excluding a signal 3.6 times larger than predicted by the SM. Performing a signal-plus-background fit the best fitted value for the signal strength is found to be: $\mu= 1.2 \pm 1.3$. The combination with a complementary search, analyzing the dileptonic channel, allows excluding a signal 3.4 times larger than the SM prediction and yields a best fitted value of: $\mu  = 1.5 \pm 1.1$. This analysis has been submitted to EPJC and represents the single most sensitive analysis to date in the search for \ttH.

A search for vector-like top partners and four-top-quark production is presented, probing several models that predict such signatures. The analysis of events with high jet and $b$-tag multiplicity, as well as multiple high-\pt\ objects allows increasing the sensitivity of the search. No excess over the background expectation is found and \unit[95]{\%} CL upper limits are set in different models. Vector-like singlets with masses below \unit[765]{\gev} are excluded, as well as vector-like doublets with masses below \unit[855]{\gev}. In the more general scenario when assumptions about the branching ratio are dropped, a vector-like top partner with a mass below \unit[515]{\gev} is excluded for any value of the branching ratio. 
Further searches for vector-like quarks have been performed in ATLAS, covering the full branching ratio plane.
The combination of this analysis with an analysis targeting the decay through a $W$ boson and a $b$-quark allows improving the sensitivity in certain regions of the branching ratio plane,
extending the exclusion limit for a singlet vector-like top quark to masses below \unit[800]{\gev}.
%With the combined analysis, a singlet vector-like top quark with masses below \unit[800]{\gev} can be excluded.
%A complementary analysis has been performed in ATLAS, and the combination 
No further combinations are performed, but an improved exclusion limit can be obtained taking the best limit for each branching ratio value from the different analyses. Vector-like top partners below \unit[730]{\gev} are excluded for any value of the branching ratios.

The same search is also used to establish limits on models predicting four-top-quark final states. A \xsec\ of 34 times  the SM prediction is excluded in the case of $\fourtop$ production with SM kinematics. In the scenario of $\fourtop$ production via an EFT model with a four-top contact interaction a \xsec\ of \unit[12]{fb} is excluded, which translates into $|C_{4t}|/\Lambda^2<\unit[6.6]{\tev^{-2}}$. Sgluons decaying to \ttbar\ are excluded for masses below \unit[1.06]{\tev}, as well as KK modes with masses below \unit[1.12]{\tev}.

Finally, a search for supersymmetric top-quark partners, or stop quarks, is presented, probing scenarios where traditional searches have little sensitivity. A search for the heavier stop, \stoptwo, is performed under the assumptions that the lightest stop quark \stopone\ is light and the mass difference to the neutralino is close to the top quark mass. In addition to high jet and $b$-tag multiplicity, the presence of neutralinos in the final state provides an experimental handle to suppress the background. No excess over the background expectation is found and \unit[95]{\%} CL upper limits are set for different masses in the  $m_{\st_2}$--$m_{\neut}$ plane, assuming BR$(\st_2 \to H \st_1)=1$.
Relaxing the assumption on the branching ratio, exclusion limits are set as a function
of the $\st_2$ branching ratios for representative values of the masses of $\st_2$ and $\neut$. The overlay with other analyses performed in ATLAS leads to excluding models with $(m_{\st_2}, m_{\neut}) = (500, 20)\gev$ for any value of the branching ratios.

The analyses presented in this dissertation constitute the most sensitive searches to date in their respective channels.

In 2015, the LHC will resume the data-taking and provide \pp\ collisions at \unit[13]{\TeV}, opening a new energy frontier. In this new energy regime, searches for new physics will continue to play a central role in the ATLAS physics program and especially searches for massive particles will benefit enormously from the increase in energy. 
The final state of \ttbar\ with additional heavy-flavor jets will continue to be a very sensitive probe for BSM solutions to the hierarchy problem. New sophisticated experimental techniques such as jet substructure will help increasing the sensitivity in searches for heavy objects.
Further refinements on the background prediction through NLO MC simulations matched to parton shower will be important to obtain the most accurate possible modeling of the \ttHF\ background. In addition, dedicated measurements of different \ttHF\ topologies should be performed to validate such predictions.

The experimental strategies developed in this thesis for Run I searches will serve as a stepping stone for more powerful searches during Run II, that will hopefully shed new light on how Nature operates at its most fundamental level.

