\chapter*{Introduction}
\markboth{INTRODUCTION}{}
\label{chapter:Introduction}
\addcontentsline{toc}{chapter}{Introduction}

The discovery of the Higgs boson in July 2012 by the ATLAS and CMS experiments was a milestone for high-energy physics, as the last missing piece of the Standard Model (SM) was found.
Nonetheless, the mass of the Higgs boson, comparable to the electroweak energy scale, is still a puzzle difficult to explain. Radiative corrections are expected to raise the Higgs boson mass by 16 orders of magnitude, from the electroweak scale to the Planck scale. This issue,
arising from the huge difference between the electroweak and the Planck scale is known as the hierarchy problem.
%The overwhelming size of these corrections arise from the huge difference between the electroweak and the Planck scale, what is known as the hierarchy problem.
This dissertation presents three analyses that probe the stability of the Higgs boson mass from different perspectives.

The largest contribution to the radiative corrections arises from the coupling to the top quark. The large mass of the top quark and its coupling of order unity to the Higgs boson makes it a very special particle, or the only ``natural'' one. Since its discovery at Tevatron, the top quark has been studied extensively and its properties have been measured in detail.
However, a measurement of the top-Higgs Yukawa coupling is not yet available.
%, showing in general good agreement with the SM expectations. 
%The large mass of the top quark, being the heaviest elementary particle known, makes it a good candidate to provide insight into possible new physics beyond the SM. 
The top Yukawa coupling is the only coupling to the Higgs boson that can be accessed directly, in particular through the measurement of the production \xsec\ of a Higgs boson in association with a top-antitop pair, \ttH. Its production \xsec\ is two orders of magnitude below the dominant gluon fusion process, and no evidence for this process has been observed yet. The dominant decay of the Higgs boson with a mass of \unit[125]{\gev} is through a pair of $b$-quarks, producing a final state of \ttbar\ with additional heavy-flavor jets. The first of the analyses aims to study the \ttH\ process and to measure its production rate, from which the top Yukawa coupling can be extracted. The corroboration of the SM nature of the coupling would confirm that the Higgs boson mass is subject to large corrections from the top quark, and a mechanism to restore the observed Higgs mass has to be present.

One of the proposed solutions to the hierarchy problem is the introduction of supersymmetry. The introduction of new partners for the SM particles, with spin differing by $1/2$, would cancel the radiative contributions to the Higgs mass, giving an explanation for its value at the electroweak scale. At the same time, supersymmetric models can provide a good candidate for dark matter. Bosonic top-quark partners have been extensively searched for at the LHC, and although a wide range of the allowed masses for supersymmetric partners' masses was excluded, some low-mass regions remain uncovered. A search for bosonic top partners is presented targeting one of the ``gaps'' where supersymmetric particles have not been excluded. 
%Pair-production of top partners and the subsequent decay to a Higgs boson and a top quark produce the targeted final state, after the decay of the Higgs boson to a \bbbar\ pair.

Although supersymmetry is a very elegant way of addressing the hierarchy problem, it is definitely not the only one. Non-supersymmetric extensions of the SM provide different ways of addressing the hierarchy problem. Some of the options are the introduction of additional dimensions, compositeness or new strong sectors. A common feature arising from such models is the prediction of fermionic top partners, which also stabilize the Higgs boson mass. Such partners, or vector-like quarks, can decay through flavor-changing-neutral-currents into a top quark and a Higgs boson. 
Another signature that can arise from these models is the production of four-top quark final states. The decay of this \fourtop\ state produces a spectacular signature that is rarely produced in the SM.

The production of a top-quark pair with additional heavy-flavor jets is a promising final state where several models of new physics, that provide solutions to the hierarchy problem, predict an enhancement. This is the final state targeted by the analyses in this dissertation given its sensitivity to the models under scrutiny.
%The \ttH\ process, and pair-production of top partners and the subsequent decay to a Higgs boson and a top quark produce the targeted final state, after the decay of the Higgs boson to a \bbbar\ pair.

The chosen final state is a very challenging one, where SM backgrounds have large uncertainties. A significant effort has been devoted to studying the modeling of \ttbb\ production, which is the main irreducible background. 
Measurements of the overall \xsec\ have been performed for this background, but no differential measurement. Therefore, the analyses have to rely on Monte Carlo simulation for the description of the \ttbb\ process.
The computation of such $2 \to 4$ process, with massive and colored partons in the final state is incredibly challenging, and predictions at NLO accuracy are essential to reduce the large scale uncertainties.

In order to reduce the impact of the systematic uncertainties, both theoretical and experimental, the analyses use a profile likelihood fit exploiting high-statistics control regions to constrain in-situ the leading uncertainties and improve the background modeling.
A detailed statistical analysis is performed in order to test for the presence of a signal in the observed data.

The results presented in this dissertation have led to the following publications by the ATLAS Collaboration:
\begin{itemize}
  \item \textit{Search for the Standard Model Higgs boson produced in association with top quarks in proton-proton collisions at $\sqrt{s}= \unit[7]{\tev}$ using the ATLAS detector.} ATLAS Collaboration, \href{https://cds.cern.ch/record/1478423}{ATLAS-CONF-2012-135}.
  \item \textit{Search for the Standard Model Higgs boson produced in association with top quarks and decaying into \bbbar\ in \pp\ collisions at $\sqrt{s}= \unit[8]{\tev}$ with the ATLAS detector.} ATLAS Collaboration, \href{http://dx.doi.org/10.1140/epjc/s10052-015-3543-1}{EPJC 75 (2015) 349}.
  \item \textit{Search for production of vector-like quark pairs and of four top quarks in the lepton plus jets final state in \pp\ collisions at $\sqrt{s}= \unit[8]{\tev}$ with the ATLAS detector.} ATLAS Collaboration, \href{http://dx.doi.org/10.1007/JHEP08(2015)105}{JHEP 08 (2015) 105}.
  \item \textit{ATLAS Run 1 searches for direct pair production of third-generation squarks at the Large Hadron Collider.} ATLAS Collaboration. \href{http://dx.doi.org/10.1140/epjc/s10052-015-3726-9}{EPJC 75 (2015) 510}.
\end{itemize}

The author has also contributed to performance studies of the hadronic calorimeter, and the work (see appendix~\ref{chapter:TileTimingPerformance}) has been documented in the following internal note:
\begin{itemize}
  \item \textit{Timing performance of the Tile calorimeter in 2011 collision data.} ATLAS Collaboration, \href{https://cds.cern.ch/record/1473262}{ATL-TILECAL-INT-2012-005}.
\end{itemize}
