\documentclass{article}
\usepackage[margin=1.3in]{geometry} %1.3 looks nice
\usepackage[breaklinks=true]{hyperref}  % working URLs
\usepackage{units}     % nice 7 TeV as $\unit[7]{TeV}$
\newcommand\fourtop{{\ensuremath{t\bar{t}t\bar{t}}}}
\def\ttH{\ensuremath{t\bar{t}H}}
\def\ttbar{\ensuremath{t\bar{t}}}
\def\bbbar{\ensuremath{b\bar{b}}}
\def\ttbb{\ensuremath{t\bar{t}+b\bar{b}}}
\def\xsec{cross section} % by Estel
\def\stoptwo{\ensuremath{\tilde{t}_2}}
\def\stopone{\ensuremath{\tilde{t}_1}}
\def\pt{\ensuremath{p_{\mathrm{T}}}} % Subscript roman not italic (EE)
\def\gev{\ifmmode {\mathrm{\ Ge\kern -0.1em V}}\else
                   \textrm{Ge\kern -0.1em V}\fi}%
\def\tev{\ifmmode {\mathrm{\ Te\kern -0.1em V}}\else
                   \textrm{Te\kern -0.1em V}\fi}%
\def\pp{\ensuremath{pp}}
\def\neutralino{\ensuremath{\mathchoice%
      {\displaystyle\raise.4ex\hbox{$\displaystyle\tilde\chi^0_1$}}%
         {\textstyle\raise.4ex\hbox{$\textstyle\tilde\chi^0_1$}}%
       {\scriptstyle\raise.3ex\hbox{$\scriptstyle\tilde\chi^0_1$}}%
 {\scriptscriptstyle\raise.3ex\hbox{$\scriptscriptstyle\tilde\chi^0_1$}}}}
\begin{document}
\section*{\textit{Javier Montejo Berlingen}}
\section*{Search for new physics in \ttbar\ final states with additional heavy-flavor jets with the ATLAS detector}

This dissertation presents searches in \ttbar\ final states with additional heavy-flavor jets using $\unit[20.3]{fb^{-1}}$ of \pp\ collision data at 
$\sqrt{s} = \unit[8]{\tev}$, recorded with the ATLAS experiment at the LHC. 
Exploiting this final state, three analyses are presented that address the
instability of the Higgs boson mass from different perspectives.

The main challenge for the presented analyses lies in the precise modeling of the background, in particular \ttbb. Since no measurements have been performed yet on the \ttbar\ production with additional heavy-flavor jets, the analyses have to rely on Monte Carlo (MC) simulation for the background. Recent developments in MC simulation have improved the description of the background, and a great effort is invested in porting the state-of-the-art predictions into the analyses. The systematic uncertainties on the modeling of the \ttbb\ background constitute the main source of sensitivity degradation.
In order to reduce the impact of the systematic uncertainties, the analyses use a profile likelihood fit to reduce in-situ the leading uncertainties. A detailed statistical analysis is performed in order to test for the presence of a signal in the observed data.

The first of the analyses aims to study the \ttH\ process and to measure its production rate, from which the top Yukawa coupling can be extracted. Neural networks are used to discriminate the \ttH\ signal from the background, where the most discriminant variable stems from the matrix element method.
No evidence for the \ttH\ process is found, and a \unit[95]{\%} confidence level (CL) upper limit is set, excluding a signal 3.6 times larger than predicted by the SM. Performing a signal-plus-background fit the best fitted value for the signal strength is found to be: $\mu= 1.2 \pm 1.3$. 

A search for vector-like top partners and four-top-quark production is presented, addressing several models that predict such signatures. The analysis of events with high jet and $b$-tag multiplicity, as well as multiple high-\pt\ objects allows increasing the sensitivity of the search. No excess over the background expectation is found and \unit[95]{\%} CL upper limits are set in different models. 
The same search is also used to establish limits on models predicting four-top-quark final states. Exclusion limits are set on SM $\fourtop$ production, $\fourtop$ production via an effective field theory model with a four-top contact interaction, sgluon pair production and Kaluza-Klein modes production. 

As last, a search for bosonic top partners, or stops, is presented, addressing supersymmetric models where traditional searches have little sensitivity. A search for the heavier stop, \stoptwo, is performed targeting models where the \stopone\ is light and the mass difference to the neutralino is close to the top mass. No excess over the background expectation is found and \unit[95]{\%} CL upper limits are set for different masses in the  $m_{\stoptwo}-m_{\neutralino}$ plane. For representative values of the masses exclusion limits are set as a function
of the $\stoptwo$ branching ratios.

The analyses presented constitute the most sensitive searches to date in their respective channels.
\end{document}
