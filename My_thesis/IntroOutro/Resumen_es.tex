\documentclass{article}
\usepackage[spanish]{babel}
\usepackage[margin=1.3in]{geometry} %1.3 looks nice
\usepackage[breaklinks=true]{hyperref}  % working URLs
\usepackage{units}     % nice 7 TeV as $\unit[7]{TeV}$
\newcommand\fourtop{{\ensuremath{t\bar{t}t\bar{t}}}}
\def\ttH{\ensuremath{t\bar{t}H}}
\def\ttbar{\ensuremath{t\bar{t}}}
\def\bbbar{\ensuremath{b\bar{b}}}
\def\ttbb{\ensuremath{t\bar{t}+b\bar{b}}}
\def\xsec{cross section} % by Estel
\def\stoptwo{\ensuremath{\tilde{t}_2}}
\def\stopone{\ensuremath{\tilde{t}_1}}
\def\pt{\ensuremath{p_{\mathrm{T}}}} % Subscript roman not italic (EE)
\def\gev{\ifmmode {\mathrm{\ Ge\kern -0.1em V}}\else
                   \textrm{Ge\kern -0.1em V}\fi}%
\def\tev{\ifmmode {\mathrm{\ Te\kern -0.1em V}}\else
                   \textrm{Te\kern -0.1em V}\fi}%
\def\pp{\ensuremath{pp}}
\def\neutralino{\ensuremath{\mathchoice%
      {\displaystyle\raise.4ex\hbox{$\displaystyle\tilde\chi^0_1$}}%
         {\textstyle\raise.4ex\hbox{$\textstyle\tilde\chi^0_1$}}%
       {\scriptstyle\raise.3ex\hbox{$\scriptstyle\tilde\chi^0_1$}}%
 {\scriptscriptstyle\raise.3ex\hbox{$\scriptscriptstyle\tilde\chi^0_1$}}}}
\begin{document}
\section*{\textit{Javier Montejo Berlingen}}
\section*{B\'usqueda de nueva f\'isica en estados finales \ttbar\ con jets adicionales pesados con el detector ATLAS}

Esta t\'esis presenta b\'usquedas de nueva f\'isica en estados finales \ttbar\ con
jets adicionales pesados usando $\unit[20.3]{fb^{-1}}$ de datos provenientes
de colisiones de \pp\ a una energ\'ia de $\sqrt{s} = \unit[8]{\tev}$, recogidos
en el experimento ATLAS en el LHC.
Haciendo uso de este estado final, tres an\'alisis han sido realizados para
investigar la inestabilidad de la masa del bos\'on de Higgs desde diferentes
perspectivas. 

La mayor dificultad para los an\'alisis es conseguir una predicci\'on precisa del
background, en concreto \ttbb.
Dado que a\'un no se han realizados medidas de la producci\'on de \ttbar\ con jets
adicionales pesados, los an\'alisis basan la descripci\'on de este proceso en
simulaci\'on mediante Monte Carlo (MC).
Avances recientes en la simulaci\'on de MC han conseguido mejorar la descripci\'on
del background, y se ha dedicado mucho trabajo a conseguir incluir estas
mejoras en el an\'alisis.
Los sistem\'aticos debidos al modelado de \ttbb\ constituyen la mayor fuente de
degradaci\'on de la sensiblidad de los an\'alisis.

A fin de reducir el impacto de los sistem\'aticos, el an\'alisis usa un ajuste a
los datos para constre\~nir in-situ los sistem\'aticos m\'as importantes.
Un an\'alisis estad\'istico detallado es necesario para probar la existencia de
una se\~nal en los datos.

El primero de los an\'alisis estudia el proceso \ttH\ para medir su producci\'on,
de lo cual se puede extraer el acoplo entre el quark top y el bos\'on de Higgs.
Para distinguir la se\~nal \ttH\ del background se usan redes neuronales, y la
variable m\'as discriminante viene de aplicar el \textit{matrix element method}.
No se observa evidencia para el proceso \ttH, y por tanto se pueden establecer
limites con una confianza del \unit[95]{\%}, excluyendo una se\~nal 3.6 veces
m\'as grande que la predicci\'on del model est\'andar. Realizando un ajuste con la
hip\'otesis de se\~nal, el mejor valor para la normalizaci\'on de la se\~nal es: $\mu=
1.2 \pm 1.3$.

Tambi\'en se ha realizado una b\'usqueda de quarks vectoriales emparejados con el
top, para estudiar varios modelos que predicen esta se\~nal.
El an\'alisis de eventos con un n\'umero alto de jets y $b$-jets, y al mismo
tiempo varios objetos con alto \pt\ permiten mejorar la sensibilidad de la
b\'usqueda. No se observa un exceso sobre la predicci\'on y se
han fijado l\'imites en varios modelos.
El mismo an\'alisis se ha usado tambi\'en para establecer l\'imites en modelos que
predicen estados finales con cuatro quarks top. Se han establecido l\'imites en
la producci\'on de cuatro tops en el model est\'andar, en un modelo efectivo con
una interacci\'on de contacto, la producci\'on de pares de sgluons y producci\'on de
modos de Kaluza-Klein en un modelos con dos dimensiones extra.

Por \'ultimo, la tesis incluye tambi\'en una b\'usqueda de tops bos\'onicos, o stops,
a fin de estudiar modelos supersim\'etricos donde las b\'usquedas t\'ipicas no
tienen sensibilidad. El an\'alisis busca el stop pesado, \stoptwo, en modelos
donde el \stopone\ es ligero, y la diferencia de masas con el neutralino es
parecida a la masa del quark top. No se observa un exceso sobre la predicci\'on
de eventos y se han establecido l\'imites en el plano de masas
$m_{\stoptwo}-m_{\neutralino}$.
Para valores concretos de la masa se han establecido tambi\'en l\'imites en
funci\'on de la fracci\'on de decaimiento del  \stoptwo.

Los an\'alisis presentados en esta tesis constituyen los an\'alisis m\'as sensibles
en los respectivos canales. 
\end{document}
